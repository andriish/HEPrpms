\documentclass[10pt]{article} \usepackage{dina4}
\usepackage{epsfig,amsmath} \usepackage{graphics}
\usepackage{cite,mcite}
 
%
% Use always these additional commands:
% ------------------------------------
\parskip 2mm plus 2mm minus 1mm     % Add space between paragraphs
\renewcommand{\topfraction}{1.0}    % These three commands assure that floats
\renewcommand{\bottomfraction}{1.0} %  (figures, tables) can cover a whole page
\renewcommand{\textfraction}{0.0}   %  and no text is required
\newcommand{\gap}{\stackrel{>}{\sim}}
\newcommand{\lap}{\stackrel{<}{\sim}}
\newcommand{\fem}{$F_2^{em}$}
\newcommand{\alps}{\alpha_s}
\newcommand{\sqrts}{$\sqrt{s}$}
\newcommand{\PT}{p_{\perp}}
\newcommand{\JP}{J/\psi}
\newcommand{\GP}{\gamma p  \rightarrow }
\newcommand{\Gg}{\gamma g  \rightarrow }
\newcommand{\sh}{\hat{s}}
\renewcommand{\th}{\hat{t}}
\newcommand{\uh}{\hat{u}}
\newcommand{\MP}{m_{J/\psi}}
\newcommand{\PO}{\ensuremath{ I\! \! P} }
\newcommand{\pom}{I\!\!P}
\newcommand{\xpom}{x_{\pom}}
\newcommand{\RE}{\em I\! \! R }
\newcommand{\CO}{\em \Pi}
\newcommand{\SMALLXC}{SMALLXa,SMALLXb}
\newcommand{\CCFM}{CCFMa,CCFMb,CCFMc,CCFMd}
\newcommand{\BFKL}{BFKLa,BFKLb,BFKLc}
\newcommand{\LDC}{LDCa,LDCb,LDCc,LDCd}
\newcommand{\alphasb}{\bar{\alpha}_s}
\newcommand{\LHAPDFLIB}{Whalley:2005nh}
%\newcommand{\JETSET}{Jetseta,Jetsetb,Jetsetc}
\newcommand{\JETSETMC}{\PYTHIAMC}
\newcommand{\LEPTOMC}{Ingelman_LEPTO65}
\newcommand{\PYTHIAMC}{pythia64}
\newcommand{\RAPGAPMC}{RAPGAP32}
\newcommand{\DGLAP}{DGLAPa,DGLAPb,DGLAPc,DGLAPd}
\def\CASCADE{{\sc Cascade}}
\def\SMALLX{{\sc Smallx}}
\def\RAPGAP{{\sc Rapgap}}
\def\LEPTO{{\sc Lepto}}
\def\PYTHIA{{\sc Pythia}}
\def\JETSET{{\sc Jetset}}
\def\HERWIG{{\sc Herwig}}
\def\HZTOOL{{\sc HZtool}}

% macro for new list environment; defl
  \newcommand{\deflab}[1]{#1\hfil}%
  \newenvironment{defl}[1]%
  {\begin{list}{}{\settowidth{\labelwidth}{#1}%
  \setlength{\leftmargin}{\labelwidth}%
  \addtolength{\leftmargin}{\labelsep}%
  \setlength{\itemsep}{0pt plus 1pt}
  \setlength{\parsep}{0pt plus 1pt}
  \setlength{\topsep}{0pt plus 1pt}
  \setlength{\partopsep}{0pt plus 1pt}
  \setlength{\parskip}{2mm plus 1mm minus 1mm}
  \let\makelabel\deflab}}%
  {\end{list}}



\makeindex
\begin{document}

%\pagestyle{empty}   % for proceedings
\thispagestyle{empty}   % to suppress the page number on the first page
%\noindent
%DESY 93--182         \hfill ISSN 0418--8993 \\
\title{ \begin{center}
{\bf  The \RAPGAP\ Monte Carlo  }\\
       version 3.2  
   \end{center}}
\author{H. Jung\\
    \vspace{0.5cm} DESY, Notkestr.85, 22603 Hamburg, FRG}
\date{\today}
\maketitle
\begin{abstract}
The physics behind the Monte Carlo event generator RAPGAP are
discussed, which includes
deep inelastic scattering, non - diffraction, diffraction and 
$\pi$ - exchange as well as resolved virtual processes.
\par
A detailed program description is given, with emphasis on parameters
the user wants to change and common block variables which 
completely specify the generated events. Subroutines for initial state parton
showers, remnant treatment and structure functions developed for \LEPTO\ and
\PYTHIA\ have been copied and modified for the use in RAPGAP.
\end{abstract}
\section{Tabular Summary}
\begin{tabular}{l l}
 program name & RAPGAP \\
 version      & 3.20/beta    \\
 date of latest version& Sept 2009 \\
 author      &  Hannes Jung (Hannes.Jung@desy.de)\\
 program size & $\sim$ 12000 lines of code\\
 input files needed & none          \\
 computer types & any with standard Fortran 77, tested on 
                 SGI, HP-UX, SUN, PC\\
 operating systems & Unix, Linux\\
 applicability & Deep Inelastic $ep$ Scattering \\
               & Deep Inelastic $ep$ Scattering with resolved photons\\
               & Deep Inelastic Diffractive $ep$ Scattering \\
               & Deep Inelastic $ep$ Scattering with $\pi$ exchange\\
               & Photo-production in $ep$ Scattering \\
               & Diffractive photo-production in $ep$ Scattering \\
               & Photo-production in $ep$ Scattering with $\pi$ exchange\\
               & hard scattering in $pp$ processes:  diffractive and
		   non-diffractive\\
  hard sub-processes included
  &  $  e q \rightarrow  e' q' $\\
  &  $  e q \rightarrow  e' q g $\\
  &  $  e g \rightarrow  e' q \bar{q} $\\
  &  $  e g \rightarrow  e' c \bar{c} $\\
  &  $  e g \rightarrow  e' b \bar{b} $\\
  &  $  \gamma q \rightarrow  q g $\\
  &  $  \gamma q \rightarrow  q \gamma $\\
  &  $  \gamma g \rightarrow  q \bar{q} $\\
  &  $  \gamma g \rightarrow  c \bar{c} $\\
  &  $ g g \rightarrow gg $ \\
  &  $ q g \rightarrow q g $\\
  &  $ q \bar{q} \rightarrow g g $\\
  &  $ q \bar{q} \rightarrow q \bar{q} $\\
  &  $ q q \rightarrow q q $\\
  &  $ q g \rightarrow q \gamma $\\
  & $ q \bar{q} \rightarrow \gamma g $\\
 QCD cascade & initial and final state parton shower or   \\
             & ARIADNE \\
 initial QED radiation & via HERACLES \\
 fragmentation model & LUND string \\
 other programs called & HERACLES 4.6.6\\
				& SOPHIA \\
                       & \PYTHIA\ 6.4 or higher \\
                       & ARIADNE 4.12  \\
                       & BASES 5.1 \\
			     & LHAPDF lib \\
availability  & \verb+http://www.desy.de/~jung/rapgap.html+ \\
\end{tabular}
\newpage

\section{Introduction}
 
In high energy physics Monte Carlo Event Generators are heavily used to
compare experimental  data with theoretical predictions. 
For example in QCD the interaction between 
quarks and gluons can be calculated in leading - or next-to-leading
order in the strong coupling constant $\alpha_s$.
In experiments only 
stable particles are measured, but 
not partons (quarks or gluons), which
cannot be described by perturbation theory, because the coupling constant
$\alpha_s$ becomes large at scales of the order of the mass of hadrons. 
Thus the hadronization has to be described by phenomenological procedures,
for example with the hadronization packages 
\JETSET\ ~\cite{\PYTHIAMC} 
or \HERWIG\ ~\cite{Herwig}.
\par
A typical Monte Carlo Event Generator usually starts by generating the
momenta of the partons involved in the interaction according to
a theoretical prescription. Such a typical event record could look like
the following (from the \PYTHIA event record):
{\small \begin{verbatim}

                           Event listing (summary)

   I  particle/jet KS     KF orig    p_x      p_y      p_z       E        m

   1  !e-!         21     11    0    0.000    0.000  156.843  156.843    0.001
   2  !p+!         21   2212    0    0.000    0.000 -156.843  156.846    0.938
   3  !gamma!      21     22    1    1.819   18.493    4.371    3.243  -18.812
   4  e-            1     11    1   -1.819  -18.493  152.472  153.601    0.001
   5  !u!          21      2    2    0.000    0.000  -23.242   23.242    0.000
   6  u         A   2      2    5    1.820   18.496  -18.874   26.488    0.006
   7  ud_0      V   1   2101    2    0.000   -0.003 -133.599  133.600    0.579
                  sum:  0.00         0.000    0.000    0.000  313.689  313.689

\end{verbatim} }
The process described by this event record is deep inelastic electron proton
scattering in the electron proton center of mass frame. The first two lines
show the beam particles. A status code KS indicates whether the 
particle (or parton) is kept only for documentation (\verb+KS = 21+) or 
 whether it is
a final state particle (parton) (\texttt{KS = 1}).
 The flavor code KF identifies 
uniquely the particle/parton. The row \verb+orig+ shows the line number of 
mother particle/parton. The third line in this example gives the 
exchanged virtual photon  
 and the fourth line  the scattered electron.
 In line 5 the parton of the proton struck by
the virtual photon is given. Line 6 
describes the scattered quark and in line 7
the remnant of the proton is given.
\par
The partons in line 6 and 7 hadronize into visible particles. Of course 
the incoming parton (line 5) could be a result of QCD radiation 
from another parton with larger energy. Also the scattered parton (line 6)
could radiate further partons before the actual hadronization takes place.
Such an example is shown in the following, where both the initial and 
scattered partons are the result of further QCD radiation.
\newpage
{\small \begin{verbatim}

                           Event listing (summary)

   I  particle/jet KS     KF orig    p_x      p_y      p_z       E        m

   1  !e-!         21     11    0    0.000    0.000 -156.843  156.843    0.001
   2  !p+!         21   2212    0    0.000    0.000  156.843  156.846    0.938
   3  !gamma!      21     22    1    1.819   18.493   -4.371    3.243  -18.812
   4  e-            1     11    1   -1.819  -18.493 -152.472  153.601    0.001
   5  !u!          21      2    2   -0.358    0.403   81.763   81.764    0.000
   6  !u!          21      2    5   -0.069    0.246   28.064   27.754   -4.167
   7  !u!          21      2    4    1.797   19.188   21.113   28.585    0.006
   8  u         A   2      2    0   -1.648    8.569    8.978   12.520    0.006
   9  g         I   2     21    0    1.795    7.372    8.854   11.660    0.000
  10  g         I   2     21    0    0.220    1.524    2.027    2.545    0.000
  11  g         I   2     21    0    1.383    1.273    3.833    4.269    0.000
  12  g         I   2     21    0    0.506    0.921    1.696    1.995    0.000
  13  g         I   2     21    0   -0.559   -0.917   51.687   51.699    0.000
  14  ud_0      V   1   2101    2    0.122   -0.249   75.397   75.400    0.579
                  sum:  0.00         0.000    0.000   -0.001  313.689  313.689
\end{verbatim} }
In this example the first 4 lines are the same as above, but line 5 now 
shows the momentum of the parton originating 
from the proton before any QCD radiation,
and line 6 gives the parton struck by the virtual photon 
(after initial state
QCD radiation). Another difference to the first example is that  a small
primordial transverse momentum has been  added 
to the parton originating from the
proton. In line 7  the scattered parton is shown before any
final state QCD radiation, whereas in line 8 the final parton after 
radiation is given. The partons radiated both from initial state and final
 state are given in the lines following line 8.
\par
More details on how to generate events according to theoretical distributions,
the basic physics processes and the way higher order QCD processes are
simulated via parton shower cascades are given in the following sections.
  

\section{Basics for Monte Carlo Generators}

In Monte Carlo Event Generators one usually wants to calculate the cross 
section for various processes with the possibility to impose experimental
cuts and to generate events according to theoretical distributions. These
two subjects are closely related to each other.
\par
First let us consider the integration of a function $f(x)$ with a Monte Carlo
method. Having two independent random numbers, say $R_1$ and $R_2$, 
generated uniformly in the interval $(0,1)$, 
then we can calculate $x_R$ with
$x_R = x_{min} + R_1 \cdot (x_{max} - x_{min})$ and $f_R$ with 
 $f_R =  R_1 \cdot f_{max}$. 
The integral $\int _{x_{min}} ^{x_{max}} f(x) dx $
can be approximated by the sum of $\sum f_R $ for which $f_R < f(x_R)$.
However, if the function $f(x)$ is strongly peeked at some value of $x$,
then this method becomes rather inefficient, because most of the
time we will have $f_R > f(x_R)$ and thus the pair of random numbers
has to be rejected.
\par
The efficiency will increase, if we can generate $x$ values according to
some approximation of $f(x)$  
 by another function $g(x)$ which is much
simpler and analytically integrable. Now we generate $x$ values according 
to the function $g(x)$. This is done by the following:
$$
 R_1 \int _ {x_{min}} ^{x_{max}} g(x) dx = 
\int _ {x} ^{x_{max}} g(x) dx
$$
Thus if we can integrate $g(x)$ analytically 
then the equation can be solved for $x$
and we have then $x$ generated according to $g(x)$. Let us assume $f(x)$
can be approximated by a function $g(x) = 1/x$. Then we have:
$$ R_1 \int _ {x_{min}} ^{x_{max}} g(x) dx 
= R_1 \log \frac{x_{max}}{x_{min}} 
 = \log \frac{x_R}{x_{min}} $$
Solving this equation 
we obtain $x_R= x_{min} ( \frac{x_{max}}{x_{min}})^{R_1}$.
In order to calculate the integral we proceed as follows:
$$
\int f(x) dx = \int \frac{f(x)}{g(x)} g(x) dx 
\simeq \sum _i  \frac{f(x_{R_i})}{g(x_{R_i})} \int g(x) dx $$
Because of the choice $f(x) \sim g(x)$, the ratio $\frac{f(x)}{g(x)}$ is
more or less constant and independent of $x$. In the example with $g(x)=1/x$
the integral is given by:
\begin{equation}
\int f(x) dx 
\simeq \sum _i x_{R_i} f(x_{R_i}) \log \frac{x_{max}}{x_{min}} 
\label{MCweight}
\end{equation}
which can be easily calculated. 
\par
Now to generate $x$ values according to $f(x)$ we use the same trick
as above: first generate $x$ according to the simple function $g(x)$ 
and then reject those values of $x$ for which $\frac{f(x)}{g(x)} > R_2$.
With this procedure $x$ values are generated according to the function
$f(x)$. Moreover this scheme is easily extended to the case of 
more than one dimension: for each variable $x_1,x_2,\ldots,x_n$ the same
procedure is applied.
\par
In \RAPGAP\ the BASES \cite{bases} integration package is used. 
This package performs a Monte Carlo integration and according to a given 
function $f(y_i)$, it generates the variables $y_i$ after internally 
optimizing a grid. In \RAPGAP\ the random numbers $x_i$ are 
generated with BASES but they are transformed to the variables $y_i$ 
according to the procedure described above. Actually the variables 
$y_i$ are the kinematic quantities $y$, $Q^2$, $x_g$, $\xpom$, $t$, $\ldots$ 
given in the next section. In all cases the true distribution is 
approximated by $g(y_i) = 1/y_i$ and the weighting factor 
(eq.(\ref{MCweight})) is applied. This procedure has the 
advantage of being very efficient, because BASES is used only to optimize
the normally small difference between the approximated and the true 
function. 

%%%%%%%%%%%%%%%%%%%%%%%%%%%%%%%%%%%%%%%%%%%%%%%%%%%%%%%%%%%%%%%%%%%%%%%
\section{Cross Section and Partonic Subprocesses}
%%%%%%%%%%%%%%%%%%%%%%%%%%%%%%%%%%%%%%%%%%%%%%%%%%%%%%%%%%%%%%%%%%%%%%%

%%%%%%%%%%%%%%%%%%%%%%%%%%%%%%%%%%%%%%%%%%%%%%%
\subsection{Standard Deep Inelastic Scattering}
%%%%%%%%%%%%%%%%%%%%%%%%%%%%%%%%%%%%%%%%%%%%%%%
The inclusive cross section for $e\;p$ deep inelastic scattering is given
in terms of the structure function $F_2^p(x,Q^2)$ and the 
longitudinal structure function 
$F_L^p(x,Q^2) = F_2^p(x,Q^2) - 2 x F_1^p(x,Q^2)$:
\begin{equation}
\frac{d \sigma (e p \to e' X)}{dy\,dQ^2}
= \frac{4 \pi \alpha ^2}{y Q^4} 
         \left(  \left( 1 -  y + \frac{y^2}{2} \right)  F_2^p(x,Q^2)
                   -   \frac{y^2}{2} F_L^p(x,Q^2) \right)
\end{equation}
with $x$ defined as $x = Q^2/(2P.q)$, $y=(P.q)/(P.l)$ 
and 
$Q^2 = -q^2 = - (l - l')^2$ where $P$ is the initial proton
four vector, $q = l - l'$ is the four vector of the virtual photon
with $l$ ($l'$)  the four vectors of the initial 
(scattered) electron.
\par
In the QCD improved  quark - parton model for
deep inelastic scattering the structure function
$F_2^p$ is given in terms of the corresponding parton distribution
functions:
\begin{equation}
F_2^p(x,Q^2) = \sum_f e^2_f \left( xq_f(x,Q^2) + x\bar{q}_f(x,Q^2)
                    \right)
\end{equation}
where the sum runs over all quark flavors and $q_f(x,Q^2)$
($\bar{q}_f(x,Q^2)$) gives the
probability of finding a quark (anti-quark) 
of flavor $f$ with a momentum fraction $x$ 
of the initial proton momentum.
\par
Different parameterization of the
parton densities in the proton can be selected in \RAPGAP\ 
(via \verb+MSTP(51)+) and the LHAPDFLIB~\cite{\LHAPDFLIB} can be accessed. 
%%%%%%%%%%%%%%%%%%%%%%%%%%%%%%%%%%%%%%%%%%%%%%%%%%
\subsection{Deep Inelastic Diffractive Scattering and pion exchange}
\label{sec:diff}
%%%%%%%%%%%%%%%%%%%%%%%%%%%%%%%%%%%%%%%%%%%%%%%%%%
In diffractive scattering (or pion exchange scattering) the
inclusive cross section is given by \cite{Ingelman_Prytz}:    
\begin{equation}
\frac{d^4 \sigma (e p \to e' X p')}{dy\,dQ^2\,dx_{\PO}\,dt}
= \frac{4 \pi \alpha ^2}{y Q^4} 
   \left(   \left( 1 -  y + \frac{y^2}{2} \right) 
                F_2^{D(4)}(x,Q^2;x_{\PO},t)
                   -   \frac{y^2}{2} F_L^{D(4)}(x,Q^2;x_{\PO},t) \right)
\end{equation}
where the diffractive structure functions $F_2^D(x,Q^2;\xpom,t)$ and
$F_L^D(x,Q^2;\xpom,t)$ are introduced, depending now on four 
variables, $x$, $Q^2$ defined as above and $\xpom=(q.p_{\PO})/
(q.P)$ and $t=(P - P')$ with $P'$ being the four 
momentum of the elastically scattered outgoing proton. In the case of
pion exchange (instead of pomeron exchange) one simply has to 
replace $F_{2;\,L}^D$
 by the corresponding $F_{2;\,L}^{\pi}$ and $\xpom$ by $x_{\pi}$.
\par
The interpretation of the diffractive structure function $F_2^D$ in terms
of parton distribution functions, in analogy  to the proton 
structure function, is 
not that clear and different approaches exist \cite{diff_ws_sum}:
\begin{itemize}

\item {\bf Resolved pomeron a la Ingelman and Schlein} 

In the model of Ingelman and Schlein \cite{IS}
diffractive scattering is described in terms of pomeron $\PO$ exchange, where
the pomeron has a partonic structure. The structure function
$F_2^D$ is then given as the product
of the probability for finding a pomeron $f_{p\;\PO}$ and the
structure function $F_2^{\PO}$ of the pomeron:
\begin{equation}
F_2^{D(4)}(\beta,Q^2;\xpom,t) = f_{p\;\PO}(\xpom,t)  F_2^{\PO}(\beta,Q^2)
\label{F2Ddef}
\end{equation}
with $\beta=Q^2/(2q.p_{\PO})$. 
In analogy to the quark - parton - model of the proton
$\beta$ can be interpreted as the momentum fraction of the total
pomeron momentum carried by the struck quark and
$ F_2^{\PO}(\beta,Q^2)$  can be seen as the
quark probabilities in the
pomeron.
Different parameterization of the 
parton densities of the pomeron can be selected (\verb+NG+ with details
given in the program description). 
The probability for finding a pomeron in the 
proton is available in three different parameterization.
\par
The parameterization of Streng and Berger et al. 
\cite{Strenga,Strengb,Berger_Collins}
 can be obtained with  \verb+NPOM=0+:
\begin{eqnarray}
f_{p\,\PO} (\xpom,t) & = & \frac{\beta^2_{p \, \PO} (0) }{16 \pi}
 \xpom^{1 - 2 \alpha_{\PO} (t)} e^{-b_0 |t|} 
\end{eqnarray}
with $\beta^2 (0) = 58.74$ GeV$^2$  and 
$\alpha_{\PO} =  \alpha_{\PO}(0) + \alpha'_{\PO} t$ and
$\alpha_{\PO}(0) = 1 + \epsilon$ describing the pomeron trajectory 
with $\epsilon=$\verb+EPSP+, $\alpha'_{\PO}=$\verb+ALPHP+
and $b_0=$\verb+RN2+ being  free parameters.
\par
The parameterization of Ingelman and Bruni 
\cite{Ingelman_Brunia,Ingelman_Brunib,Ingelman_Brunic}
 is given (\verb+NPOM=1+):
\begin{equation}
f _{p\,\PO} (\xpom,t) = \frac{1}{2}\frac{1}{2.3}
                             \frac{1}{\xpom}\left[6.38 e^{-8|t|}
                                      + 0.424 e^{-3|t|} \right]
\end{equation}
\par
Donnachie and Landshoff \cite{DL_84} 
give the following (via \verb+NPOM=2+) :
\begin{eqnarray}
f _{p\,\PO} (\xpom,t) & = & \frac{9 \delta ^2}{4 \pi ^2} 
       \left[F_1(t)\right]^2
                   \xpom^{1 - 2 \alpha_{\PO}(t)}
\end{eqnarray}
with $\delta^2 = 3.26$ GeV$^2$ and the elastic form factor $F_1(t)$:
\begin{equation}
F(t) =        \frac{4 m_p ^2 - 2.8 t}{4 m_p ^2 - t}
                   \frac{1}{(1 - t/0.7)^2}
\end{equation}

\item {\bf Resolved pomeron with a parameterization to
H1 data} 

In \cite{H1_F2D3_97} the H1 collaboration has given a parameterization of the
diffractive structure function in terms of pomeron and meson trajectories. This 
parameterization can be selected by \verb+NG=-10,-11,-12+
for the NLO parameterizations and by \verb+NG=-13,-14,-15+
for the LO ones, according to a 
different ansatz for the parton distribution in the pomeron at the starting
scale for the $Q^2$ evolution: \verb+NG=-10+ assumes only quarks at the starting
scale $Q_0$, \verb+NG=-11+ has quarks and gluons, whereas in \verb+NG=-12+ 
a gluon distribution peaked at large values $\xi$, the parton momentum
fraction, is assumed.
\par
With \verb+NPOM=-10+  the predefined flux for pomeron exchange is selected,
\verb+NPOM=-11+ gives the meson contribution only, and \verb+NPOM=-12+ gives the
mix of pomeron and meson as obtained by the H1 collaboration \cite{H1_F2D3_97}.

In \cite{H1_F2D3_06} a new set of diffractive pdfs was obtained in NLO in the
$\bar{MS}$ scheme. They can be selected via \verb+NG=-30,-31+ for set A and set
B (respectively). The flux can be selected for pomeron only \verb+NPOM=-30+,
Reggeon (meson) only \verb+NPOM=-31+ and a combination \verb+NPOM=-32+. Please
note that the pdfs contain only light partons (i.e. gluon, u-,d-, s-quark
densities). For the contribution of heavy quarks (charm and beaty) the BGF
process via  \verb+IPRO=14+ has to be run separatly and added to the light quark
cross section.
 
\item {\bf Two gluon exchange for diffraction} 

This approach is mainly intended to describe exclusive high $p_t$
jet production, but in the model of \cite{Wusthoff} estimates on the
total inclusive diffractive cross section are given.
The calculation of diffractive di-jet production can be
performed using pQCD for large photon virtualities $Q^2$ and 
high $p_t$ of the 
$q (\bar{q})$ 
jets~\cite{Wusthoff,Diehl1,Diehl2,Bartels_dijet_ws,Bartels_jets,Bartels_asym}.
\par
The process is mediated by two gluon exchange. Different
assumptions on the nature of the exchanged gluons 
can be
made: in \cite{Diehl1,Diehl2} (\verb+NG=42+, \verb+NPOM=42+)
the gluons are non perturbative, in \cite{Wusthoff} 
(\verb+NG=40+, \verb+NPOM=40+) 
they are a hybrid of
non perturbative and perturbative ones and in 
\cite{Bartels_jets,Bartels_asym}
 (\verb+NG=41+ and \verb+NPOM=41+ or \verb+IPRO=21+,
which is done technically different and is more efficient) 
they are
taken from a NLO parameterization of the proton structure 
function \cite{GRVa,GRVb}.
The cross section is essentially proportional to the  proton gluon density 
squared: 
$\sigma \sim \left[\xpom G_p\left(\xpom,\mu^2 \right)\right]^2 $
with $\mu^2$ the scale explicitly given in the references.
\par
The perturbative calculation of $q\bar{q}$ (\verb+IPRO=21+) and 
$q\bar{q} g$ (\verb+IPRO=20+) 
(\cite{Bartels_jets,Bartels_qqga,Bartels_qqg-charm}) 
can be used for light quark and heavy
quark production. With \verb+Iqqg+ different levels of approximation 
in the calculation for \verb+IPRO=20+ can be selected. However it is recommended to use the 
full calcualtion with  \verb+Iqqg=0+.

The un-integrated gluon density needed for 
the calculation is
selected via \verb+IGLU+. Different sets are available:
 the CCFM unintegrated gluon {\bf set A}~\cite{jung-dis04}
with \verb+IGLU=1+,
a simple numerical derivative of a standard integrated gluon density
$\frac{d xg(x,Q^2)}{dQ^2}$ taken
from~\cite{GRV95}  (\verb+IGLU=2+), the one in the
approach 
of Bl\"umlein~\cite{Bluemlein} and coded 
by~\cite{baranov_zotov_1999,baranov_zotov_2000} (\verb+IGLU=3+),
the unintegrated gluon density of KMS\footnote{A. Stasto kindly
provided the program code.}~\cite{martin_stasto} 
(\verb+IGLU=4+, stored in \verb+kms.dat+), 
the one of the saturation model by~\cite{wuesthoff_golec-biernat}
 (\verb+IGLU=5+) and the one of KMR\footnote{M. Kimber kindly
provided the program code .}~\cite{martin_kimber} (\verb+IGLU=6+, 
stored in \verb+kmr.dat+).
\par
Due to the different gluon densities different $\xpom$ dependencies of
the cross sections are expected and further discussed in
\cite{Bartels_dijet_ws,Bartels_qqga,Bartels_qqg-charm}, 
where also numerical estimates are presented.

\item {\bf Semi-classical approach of Buchm\"uller, McDermott and Hebecker} 

In \cite{Buchmuller_Hebecker_Mcdermott} Buchm\"uller et al.\ define an
effective diffractive gluon density:
\begin{equation}
\xpom G^D(\xpom,\beta) = \frac{C_1}{\beta \cdot C_g - \beta + 1} 
               \cdot  \frac{1}{\xpom}
\end{equation}
with $C_1$ and $C_g$ being free parameters. In addition the dipole form 
factor for the $t$ dependence is applied. Note that with $C_g=1$ a
constant gluon density is obtained. In that approach there is only 
a $1/\xpom$ dependence in contrast to the other models, where at least 
a $1/\xpom^{1+\epsilon}$ is present.


\item {\bf Nikolaev Zakharov model for diffraction} 

A parameterization of the diffractive structure function in the model of
Nikolaev and Zakharov~\cite{NikZak} can be selected 
(\verb+NG=30+, \verb+NPOM=30+). In the implementation in \RAPGAP\
only QPM type events can be 
generated, no higher order QCD processes like QCD-C or BGF
are possible for \verb+NG=30+, \verb+NPOM=30+.

\item {\bf Saturation Model (SATRAP)} 

The saturation model SATRAP~\cite{satrap} is implemented to describe $q\bar{q}$  and 
$q\bar{q} g$  final states (\verb+IPRO=30+).

\item {\bf User defined model for diffraction} 

 The user can freely define his preferred model for diffraction by supplying
 an effective diffractive structure function 
(\verb+SUBROUTINE USDIFFR+). 
 Here no assumption about factorization is made.
\par
The basic partonic
 subprocesses available are: $eq \rightarrow e' q'$ (QPM),
                            $eq \rightarrow e' q g$ (QCD-C),
 $eg \rightarrow e' q \bar{q}$ (BGF),
  $eg \rightarrow e' c \bar{c}$ (BGF),
 $eg \rightarrow e' b \bar{b}$ (BGF), which are discussed in more detail
in the following sections.
\end{itemize}
In all cases there is the possibility to add primordial $p_T$ 
(via switch \verb+IALMKT=1+) 
according to $\exp{(-5.5 \cdot p_T^2)}$ to the
partons when process $IPRO=12$ is selected.
%%%%%%%%%%%%%%%%%%%%%%%%%%%%%%%%%%%%%%%%%%%%%%%%%%%%
\subsection{Deep Inelastic Scattering including Diffraction and Pion exchange}
%%%%%%%%%%%%%%%%%%%%%%%%%%%%%%%%%%%%%%%%%%%%%%%%%%%
Inclusive deep inelastic scattering obviously includes diffraction and 
$\pi$ exchange etc. The total cross section of DIS, $F_2$, can be 
written as a sum of diffractive and non - diffractive contributions:
\begin{equation}
F_2(x,Q^2) = F_2^{non - diff}(x,Q^2) + F_2^{diff}(x,Q^2) + \ldots
\end{equation}
where the dots indicate that other exchanges like pion exchange etc. can
be treated similarly.
\par
The diffractive contribution to $F_2$ is given by $F_2^{diff}$:
\begin{equation}
F_2^{diff}(x,Q^2) =\int F_2^{D(4)}(\xpom,t;\beta,Q^2)
      \delta(\beta\xpom - x)\; d\xpom \;d\beta\;dt
\end{equation}
With the knowledge of $F_2$ and $F_2^{diff}$ the non - diffractive
part can be calculated which is then used in the Monte Carlo to generate the
proper mix of diffractive and non - diffractive processes 
(\verb+IDISDIF=1+). Similarly non - diffraction, diffraction and 
pion - exchange are selected with \verb+IDISDIF=2+.

%%%%%%%%%%%%%%%%%%%%%%%%%%%%%%%%%%%%%%
\subsection{QED radiative corrections}
%%%%%%%%%%%%%%%%%%%%%%%%%%%%%%%%%%%%%%

 Real photon emission from both the incoming and scattered electron
can have large effects on the reconstruction of the DIS kinematic 
variables $x$, $Q^2$, $y$. 
These QED radiative effects are simulated with
the HERACLES event generator \cite{HERACLESa,HERACLESb}. 
HERACLES is used to generate the $e -  \gamma^* - e$ vertex including
photon emission from the incoming and outgoing electron as well as 
virtual corrections. The interface to HERACLES is switched 
 on with \verb+IPRO=1200(1400)+ to generate QED radiative effects for 
 processes \verb+IPRO=12(14)+.For very low masses $W$ an interface to the
 SOPHIA~\cite{sophia} generator is used.
\par
Since QED radiative effects tend to lower the actual $Q^2$ of the 
photon which is available for any subsequent process, one has to
ensure that parton densities behave well at very small $Q^2$. 
In the quark parton model the QPM process has to
vanish for $Q^2 \rightarrow 0$. On the other hand parton densities
are usually only parameterized down to a fixed value of $Q_0^2$. In
order to ensure a well defined small $Q^2$ behavior, a exponential 
suppression factor is applied: $1 - \exp(-$\verb+Q2SUPP+ $\cdot Q^2)$ 
with \verb+Q2SUPP=3.37+ being the default from HERACLES.
Values of \verb+Q2SUPP=5+ seems appropriate for parton densities 
evolved down to $Q^2_0\sim 1$ GeV$^2$ and \verb+Q2SUPP=10+ for
$Q^2_0 \sim 0.5 $ GeV$^2$.

%%%%%%%%%%%%%%%%%%%%%%%%%%%%%%%%%%%%%%%
\subsection{Order $\alpha_s$ matrix elements}
%%%%%%%%%%%%%%%%%%%%%%%%%%%%%%%%%%%%%%%

%%%%%%%%%%%%%%%%%%%%%%%%%%%%%%%%%%%%%%%
\subsubsection{Order $\alpha_s$ matrix elements with $p_T$-cutoff}
%%%%%%%%%%%%%%%%%%%%%%%%%%%%%%%%%%%%%%%

With the knowledge of $F_2$ the total cross section can be described in 
terms of scattering a virtual photon 
on a quark or anti-quark. However this quark may have
been originated from another parton, usually described by the 
$Q^2$ dependence of the parton densities resulting in a different 
hadronic final state. The process where a initial parton 
carrying a 
momentum fraction $x_i$, splits into other partons which then 
hard scatter with the photon, can be simulated in QCD parton showers 
(selected via \verb+IFPS+) , 
for incoming (\verb+IFPS = 1+), outgoing partons (\verb+IFPS=2+) and both 
(via \verb+IFPS = 3+). These QCD parton showers are based on the 
leading log DGLAP \cite{DGLAPa,DGLAPb,DGLAPc,DGLAPd} splitting functions
in leading order $\alpha_s$. 
\par
A more detailed simulation of 
leading order $\alpha_s$ processes like $\gamma^* g \to q \bar{q}$ 
(BGF, Fig.~\ref{DISmc}c.)
and $\gamma^* q \to q g$ (QCD - Compton, Fig.~\ref{DISmc}d.)
 can be obtained when 
the exact matrix elements for these processes are included.
\par
These 
processes can be simulated separately with \verb+IPRO = 13+ for 
BGF (light quarks), \verb+IPRO=14+ for BGF (heavy quarks)  
and \verb+IPRO = 15+ for QCD-C.
The prompt photon process $\gamma q \to \gamma q$ is simulated via
\verb+IPRO = 16+.
\par
A full simulation including QPM - as well as 1st order $\alpha_s$ - processes
is selected with \verb+IPRO=12+ (or \verb+IPRO =1200+ for
 HERACLES) together 
with \verb+IFULL=1+. The decision whether to generate a QPM or a 1st order 
$\alpha_s$ event is based on the cross section for a particular process 
at a given $x$ and $Q^2$. Therefore for each event the cross section
for BGF light quarks, BGF heavy quarks and QCD - Compton has to be 
obtained by numerical integration including the proper parton densities.
If the scale chosen for $\alpha_s$ and the parton densities (\verb+IQ2+) 
is $Q^2$ or $\hat{s}$, then the matrix elements are integrated 
analytically over $z$ leaving only a one dimensional numerical integration.
If, however, the scale is $p_T^2$ (or any function of it) then $\alpha_s$ and
the parton densities cannot be factorized, and a time consuming 
two dimensional numerical integration has to be performed.
As an alternative the QCD probabilities are calculated once and stored in
a grid (\verb+IGRID=1+).
 This approach is faster but less accurate. This procedure is
recommended the scale is chosen to be $p_T^2$ (\verb+IQ2=3+) or 
$Q^2 + p_T^2$ (\verb+IQ2=5+).
\par
In order to avoid divergences in the matrix elements for massless quarks
a cutoff in $p_T^2$  has to be specified 
(\verb+PT2CUT+). The minimum $p_T^2$ is only restricted by the 
requirement that the sum 
of the order $\alpha_s$ processes is smaller or equal $F_2$. 
\par
 However care has 
to be taken in the case of pomeron exchange, that $Q^2$ evolved 
parton densities are used with the proper normalizations for quark and
gluon densities, otherwise it can occur that the 
probability for photon - gluon fusion exceeds the total cross section 
calculated from $F_2^{\PO}$ with the quark densities even for 
relatively large \verb+PT2CUT+.
%%%%%%%%%%%%%%%%%%%%%%%%%%%%%%%%%%%%%%%
\subsubsection{NLO and Order $\alpha_s$ matrix elements }
%%%%%%%%%%%%%%%%%%%%%%%%%%%%%%%%%%%%%%%
The problem using a $p_T^2$ cutoff, as described in the previous subsection
can be overcome by a proper treatment of the NLO corrections to $F_2$. Such a
scheme is described in detail in \cite{Collins-collfac,Collins-collfac2}.
The LO ($\alpha_s^0$) and the NLO ($\alpha_s$) part are treated according the
$\bar{MS}$ subtraction scheme, reformulated such that it properly can be used
together with initial and final state parton showers, avoiding any double
counting \cite{sschilling}.
 When using this scheme, the NLO parton densities calculated 
in the $\bar{MS}$ scheme should be selected. The program then transforms the
parton densities from the $\bar{MS}$ to the $BS$ scheme.
\footnote{The implementation of this scheme into RAPGAP was done by S.
Schilling.}  
However, at present only the BGF part is implemented. This scheme is switches on
with \verb+IBS=1+.

%%%%%%%%%%%%%%%%%%%%%%%%%%%%%%%%%%%%
\subsection{Treatment of heavy flavor production}
%%%%%%%%%%%%%%%%%%%%%%%%%%%%%%%%%%%%
Heavy flavors, like charm and bottom, are usually produced via boson - gluon
fusion. However in the evolution of the parton densities, and similarly in the
evolution of $F_2(x,Q^2)$, heavy quarks are often treated in the same way as
light quarks, often not even respecting the production threshold properly.
Thus when calculating 1st order $\alpha_s$  matrix elements there is
a contribution of QCD-C from charm in the proton, together with BGF and
eventually also QPM type events with charm. 
Whether this is a consistent treatment, depends on the details of the $Q^2$
evolution of the parton densities.
\par
GRV \cite{GRVa,GRVb}, for example, calculate $F_2^{charm}(x,Q^2)$ only via BGF.
Thus in their approach the charm quark density is zero. However, for technical
reasons for calculation of the total cross section, in the parameterization of
the parton densities, as implemented in LHAPDFLIB \cite{\LHAPDFLIB}  a charm quark
density is provided, which is calculated from the BGF process. This is correct
as long as only the total cross section is considered, which is calculated via
\begin{equation}
\sigma ^{\gamma* \,p} (x,Q^2) = \frac{4 \pi^2 \alpha }{Q^2} \cdot F_2(x,Q^2) =
  \frac{4 \pi^2 \alpha }{Q^2} \cdot \sum_{i=-6}^6 e_i^2 \cdot xq_i(x,Q^2) 
\end{equation}
For a proper description of the hadronic final state 
using the GRV parton densities, only light quarks are
allowed for QPM and QCD-C processes, and all heavy quarks are only produced via
BGF. 
\par
Just from the parton density parameterization itself, it is nor possible to
decide whether heavy quarks are produced only via BGF or not. Thus the user has
to take care of this via \verb+NFLQCDC+, which is set by default to 
\verb+NFLQCDC=3+. If heavy quarks are allowed to be produced also by QCD-C
processes, then the user should set \verb+NFLQCDC=4+, or \verb+NFLQCDC=5+.


%%%%%%%%%%%%%%%%%%%%%%%%%%%%%%%%%%%%
\subsection{QCD processes in hardon-hadron collisions}
The hard subprocesses implemented  are:
\begin{eqnarray}
 gg \rightarrow q \bar{q} & \verb+IRPA=1+\\
 g g \rightarrow gg& \verb+IRPB=1+ \\
 q g \rightarrow q g & \verb+IRPC=1+\\ 
 q \bar{q} \rightarrow g g & \verb+IRPD=1+\\
 q \bar{q} \rightarrow q \bar{q}&\verb+IRPE=1+\\
 q q \rightarrow q q & \verb+IRPF=1+\\
% \\q q \rightarrow q q (BFKL)& \verb+IRPG=1+.
q g \rightarrow q \gamma & \verb+IRPH=1+\\ 
 q \bar{q} \rightarrow g \gamma & \verb+IRPI=1+.
\end{eqnarray}
The corresponding color configurations are given in \verb+ICOLORA+.
\subsubsection{Hadron-hadron collisions} 
In hadron hadron collisions non-diffractive (\verb+IDIR=1+) diffractive processes (\verb+IDIR=0+) with diffractive parton densities and processes with $\pi$-exchange (\verb+IDIR=0+, \verb+NG=20+) are available (for details on switches and PDFs see \ref{sec:diff}).

\subsubsection{Resolved Photons in \boldmath$\gamma p$ and DIS}
%%%%%%%%%%%%%%%%%%%%%%%%%%%%%%%%%%%%

Resolved photon processes play a important role 
in high $p_T$ jet - production in photo-production. 
Any internal
structure in the proton as well as in the photon can be resolved as long
as the scale of the hard subprocess, which is of the order of $p_T^2$,
is larger than the inverse radius of the proton 
$1/R^2_p \sim \Lambda_{QCD}^2$
 and the photon 
$1/R^2_{\gamma} \sim Q^2$.
\par
In DIS resolved photon processes \cite{Chyla_res_gamma}
can play a role when the scale $\mu^2$
 of the
hard subprocess is larger than $Q^2$, the inverse size of the photon. 
\par
Resolved photon processes in DIS 
are more complicated than in photo-production because one has treat properly
the kinematics of the scattered electron. This is done with the
Equivalent Photon Approximation, giving the flux of virtual transverse 
polarized photons \cite{\RAPGAPMC}. The structure
of the virtual photon is defined by parameterizations of the parton densities
of the virtual photon, now depending on the scale $\mu^2$ and $Q^2$:
$x_{\gamma} f_{\gamma}(x_{\gamma},\mu^2,Q^2)$ 
\cite{GRS,Sasgam,Drees_Godbole}. Resolved photon processes can be selected via
\verb+IPRO=18+ for real and virtual photons (depending on $Q^2_{min}$).
\par
Since this processes can only occur for $\mu^2 > Q^2$, large differences
in the cross section are expected when choosing different scales $\mu^2$,
 like
$\mu^2=p_t^2$, or $\mu^2=\hat{s}$. 
A part of this dependence is removed when
next-to-leading order diagrams are considered. For example the process
$qg \rightarrow qg$ also occurs in a NLO calculation for jet production
in DIS \cite{Mirkes_nlo,catani_nlo,Kramer_res_gamma}. 
However in resolved photon processes there is 
also the evolution of the virtual photon parton densities from $Q^2$ to 
the scale $\mu^2$, which is a re-summation to all orders. Therefore 
differences between full NLO calculations and resolved photon processes
are expected \cite{Kramer_res_gamma}. 
\par
From version 2.06/29 resolved photon processes are also 
implemented for diffraction and $\pi$ exchange.
%%%%%%%%%%%%%%%%%%%%%%%%%%%%%%%%%%%%
\subsection{Scales, $\alpha_s$ and Parton Distribution Functions}
%%%%%%%%%%%%%%%%%%%%%%%%%%%%%%%%%%%%

In leading order $\alpha_s$ processes
the renormalization scale $\mu_R$ ($\alpha_s (\mu_R^2)$) 
and factorization scale $\mu_F$ ($xf(x,\mu_F^2)$) 
are not well defined, and any choice of 
$\mu_R^2 = \mu_F^2 =Q^2,\;p_t^2,\;4p_t^2,
\;\hat{s},\; Q^2 + p_t^2$ is reasonable. 
There are essentially two competitive effects:
a larger scale gives smaller $\alpha_s(\mu^2)$ but
a larger parton density $xf(x,\mu^2)$ at fixed $x$.
The net effect depends on the
details of the interaction and the parton density. 
\par
\RAPGAP\ offers different choices for the scales:
$\mu^2 = m^2,\;\hat{s},\;m^2+p_t^2,\;Q^2,
\; Q^2 + p_t^2$, selected with \verb+IQ2 = 1,...,5+ (\verb+IQ2=1+ only makes 
sense for heavy quark production). A scale factor 
\verb+SCALF+ can be chosen to study the effect of different scales:
\verb+SCALF=4+ with \verb+IQ2=3+ will result in a scale $\mu^2 = 4 p_t^2 $
 for $\alpha_s$ and the parton densities. 
%%%%%%%%%%%%%%%%%%%%%
% in version 2.06/02
%%%%%%%%%%%%%%%%%%%%%
In case of
\verb+IQ2=5+ a scale factor \verb+SCALF=4+ will result in 
$\mu^2 = Q^2 + 4 \cdot p_t^2 $.
\par
Considering the $O(\alpha_s)$ matrix element processes one could argue 
that the scale corresponds to the propagator $t$ ($u$ for the crossed diagram)
\cite{MEPS},
which is 
\begin{eqnarray}
\mu^2& = & \max(|t,u|) \\
     & = & \max(|-Q^2 - 2 q.p_{f\;1},-Q^2 - 2 q.p_{f\;2}|) \\
     & = & \max(1-z,z) \cdot \left( \hat{s} + Q^2 \right)
\label{scale}
\end{eqnarray}
with $ z = \frac{p_i.p_{f\;1}}{p_i.q}$ and $p_i$ ($p_f$) being the momentum
of the initial (final) state parton. Expressing $p_T^2$ in terms of $z$ it is 
easily seen that this choice of the scale 
$\mu^2$ is always larger than $p_T^2=z(1-z) \hat{s}$.
\par
The strong coupling constant is calculated consistently in the one loop
expression (using \verb+ULALPS+ from \PYTHIA
\cite{Jetseta,Jetsetb,Jetsetc}):
$$ \alpha_s(\mu^2) = 
\frac{12 \pi}{(33-2n_f) \log(\mu^2/(\Lambda^{n_f}_{QCD})^2)} $$
Often $\Lambda_{QCD}$ is just taken from
the parameterization of the parton densities, but when using non - diffractive
processes together with diffraction and pion exchange, it is not clear which
$\Lambda_{QCD}$ to take, the one from the parameterization of the proton, or the
pomeron or the pion. The same problem occurs when resolved photon processes are
included. By default, the value for $\Lambda_{QCD}$ is taken from the parton density. Using \verb+ILAM+ it can be  specified separately by 
via the switches of \PYTHIA: \verb+MSTU(112)+ giving the No. of flavors with
respect to  $\Lambda_{QCD}$, \verb+PARU(112)+ giving the value of $\Lambda_{QCD}$.
The default value is $\Lambda_{QCD}^5 = 0.25$ and \verb+MSTU(112)=5+.
The number of actual open flavors $n_f$ depends
on $\mu^2$, and $\alpha_s$ is required to be continuous at the flavor
thresholds.
\par
Most recent parton density parameterizations of the proton and the pion can be
accessed from the LHAPDFLIB \cite{\LHAPDFLIB} with 
\verb+MSTP(51)=NSET+
(example: \verb+MSTP(51)=10150 + gives CTEQ6L NLO structure function of the 
proton). Technically \verb+PYSTFU+ from \PYTHIA\ \cite{\PYTHIAMC} has been
modified to be used within \RAPGAP . 
 If the user has no
access to LHAPDFLIB internally coded  parameterizations can be used with 
\verb+MSTP(51)<10+ with details given in the program description.
\par
The parton densities of the pion are accessed similarly with
\verb+MSTP(52)+.
\par
The parton densities of the virtual photon are accessed with
\verb+MSTP(56)+. In the program the parameterizations GRS \cite{GRS}
(\verb+MSTP(56)=1+)
and SaS~\cite{Sasgam} (\verb+MSTP(56)=2+) are available. 
For \verb+MSTP(56)>10+ again LHAPDFLIB is used for real photons and the
virtual photon suppression factor of Drees - Godbole \cite{Drees_Godbole}
is applied.
\par
In the resolved photon case a factor \verb+SCALQ2+ can be specified, which
regulates the phase space region, where resolved photon processes are
possible: \verb+SCALQ2=1+ means $\mu^2 \geq 1 \cdot Q^2$, and 
\verb+SCALQ2=10+ means $\mu^2 \geq 10 \cdot Q^2$ where $\mu^2$ is the
scale specified by \verb+IQ2+.
 
%%%%%%%%%%%%%%%%%%%%%%%%%%%%%%%%%%%%
\subsection{Exclusive Vector-meson Production}
%%%%%%%%%%%%%%%%%%%%%%%%%%%%%%%%%%%%

Vector meson production is included naturally in diffractive scattering.
This is easiest seen for $J/\psi$ production. Suppose we have a system of
a $c \bar{c}$ quark, plus eventually additional gluons in the final state.
If the invariant mass $m_x=(q + p_{\pom})^2 < 4 \cdot m^2_{D^0}$ 
then only $J/\psi$'s can be produced ($\eta_c$ production is not possible
because of spin constraints). 
Technically this is done  in the fragmentation
program \PYTHIA \cite{Jetseta,Jetsetb,Jetsetc}. 
However the ratio of 
spin 0 to spin 1 mesons has to be included by hand by 
setting \verb+PARJ(13)=1+
(for the light $u,d$ mesons \verb+PARJ(11)=1+, and for $s$ mesons 
\verb+PARJ(12)=1+) for  \PYTHIA. 
%The production of the
%light vector-mesons $\rho$, $\omega$ and $\phi$ proceeds in a similar way to
%that of the $J/\psi$, but the ratio $\rho:\omega=
%9:1$ has been included by hand by G. Briskin \cite{Briskin_vm}
%in a modified subroutine
% \verb+LUKFDI+ of \PYTHIA.
\par
Exclusive vector meson production implies certain restrictions on the
kinematic variables $\xpom$ and $\beta$:
$$
\xpom = \frac{q.p_{\pom}}{q.P}=
        \frac{Q^2 + m_x ^2}{Q^2 + W^2} $$
$$
\beta = \frac{Q^2}{2\cdot q.p_{\pom}} =
        \frac{Q^2}{Q^2 + m_x^2} $$
Thus for $m_x=m_{VM}$ and  fixed $W$ 
the variables $\xpom$ and $\beta$ depend only
$Q^2$. Thus changing $Q^2$ means moving in the two dimensional
$\xpom - \beta$ plane, which has consequences on the $Q^2$ dependence
of the $\gamma ^* p \rightarrow \mbox{VM} p$ cross section.
\par
The cross section for vector meson production of course depends 
crucially  on the underlying subprocess. Using the recent parameterization
of $F_2^{D(3)}(\xpom,\beta,Q^2)$ of the H1 collaboration \cite{H1_F2D3_97}
together with \verb+IPRO=12+ and \verb+IDIR=0+ (which is QPM on a
parton in the pomeron) $J/\psi$ production as measured by the experiments
H1 and ZEUS \cite{Jpsi_H1_dis,Jpsi_ZEUS} 
can be well described as a function of $Q^2$ and also 
$W$~\cite{Jung_Eilat}. 
Surprisingly a $\sim 1/Q^4$ dependence of the photon proton cross section
appears, which can be interpreted as a higher twist effect.  However in 
the model described here this $Q^2$ dependence appears naturally from
the $\beta$ dependence of the structure function $F_2^{D(3)}$, since as 
shown above, changing $Q^2$ is equivalent with changing $\beta$ for 
a given vector-meson.





%%%%%%%%%%%%%%%%%%%%%%%%%%%%%%%%%%%%%%%%%%%%%%%%%%%%%%%%%%%%%%%%%%%%%%%
\section{Remnant Treatment and Fragmentation}
%%%%%%%%%%%%%%%%%%%%%%%%%%%%%%%%%%%%%%%%%%%%%%%%%%%%%%%%%%%%%%%%%%%%%%%
The fragmentation in \RAPGAP\ is done with the Lund - string model as 
implemented in \PYTHIA/\PYTHIA\ \cite{\PYTHIAMC}.
 For the treatment of the 
remnant 
and initial state parton showers, subroutines from the 
programs \LEPTO\~6.1 \cite{MEPS} (\verb+LMEPS+, \verb+LPRIKT+, \verb+LREMH+)
and \PYTHIA\ \cite{\PYTHIAMC} (\verb+PYREMN+, \verb+PYSPLI+, \verb+PYSSPA+) 
have been copied 
and 
modified to be applicable here. The original structure of these subroutines
is kept.
\par
 Optionally higher order QCD radiation
can be simulated via leading log parton showers (\verb+IFPS=1+ for initial
state parton shower, \verb+IFPS=2+ for final state PS,
\verb+IFPS=3+ for
initial and final state PS) or with the color dipole model 
(as implemented in  ARIADNE~\cite{CDM} with \verb+IFPS=10+).
%\subsection{Les Houches interface} 
%Several authors of popular Monte
%Carlo and matrix element 
%programs attending the {\it Physics at TeV Colliders Workshop}
%in Les Houches, 2001 have agreed on a generic format for the transfer
%of parton level event configurations from matrix element event generators 
%(MEG) to showering and hadronization event generators (SHG) 
%\cite{LHA-generator}. With \verb+ILHA=1+ \RAPGAP\ writes the full parton level
%events to the file \verb+rapgap.gen+, which can be read in directly by the
%\PYTHIA\ and \HERWIG\ programs to perform the hadronisation. This option is best
%suited to estimate the uncertainty coming from hadronsiation correction.
%\par
%Since \HERWIG\ requires a finite gluon mass $m_{gluon} \sim 0.7$~GeV,
% \PYTHIA\ parameters need to be changed:
% \verb+PMAS(21,1)=0.7+ for the gluon mass $m_{gluon}$ 
% \verb+COMMON/PYDAT2/+ and 
% \verb+PARJ(82)=1.2+ for the invariant mass cut-off of parton showers
% (\verb+COMMON/PYDAT1/+).. 
\subsection{Hadronic final state}
In a standard inelastic scattering process
a parton carrying color is removed from the 
target (a proton and/or a photon in case of resolved photon processes)
 and in general 
 colored remnants are left. These remnants together with the
colored partons of the hard interaction  must form  color
singlet states (see Fig.~\ref{DISmc}).
\begin{figure}[ht]
\begin{center}
\centerline{\includegraphics[bb=58 442 555 756,width=1\columnwidth]{dismc.epsi}}
\end{center}
\caption{Basic processes for inelastic lepton nucleon scattering.
Indicated are the color strings and the proton remnant.
$a$. shows the lowest order process for scattering on a valence quark.
The remnant of the proton is just a di - quark-quark.
$b$. shows the lowest order process for scattering on a sea quark.
The remnant of the proton is the corresponding anti-quark , 
a valence quark and a valence di-quark.
 $c$. shows the
$O(\alpha _{em}\alpha_s)$
for gamma gluon fusion (the crossed diagram is not shown).
 The proton remnant is the valence quark and
valence di-quark.
 $d$. shows the
$O(\alpha _{em}\alpha_s)$
for QCD Compton (the crossed diagram is not shown) on a valence quark.
 The proton remnant is the same as in $a$.
 $e$. shows the
$O(\alpha _{em}\alpha_s)$
for QCD Compton (the crossed diagram is not shown) on a sea quark.
 The proton remnant is the same as in $b$.
\label{DISmc}}
\end{figure}
 Color strings connect the colored partons with the remnants and these
color strings
generate  particle
flows between the remnants and the partons of the hard
scattering.
\par 
In the lowest order process ($\gamma^* q \to q'$) the remnant is 
a $di-quark$ 
when the scattering occurs on a valence quark of the proton
(Fig.~\ref{DISmc}$a$.).
If, however, the scattering occurs on a sea quark or a anti - quark,
the remnant is treated as a valence quark and valence $di - quark$ plus
the corresponding anti - quark from the sea. This ``new" 
sea quark treatment was first developed and  
implemented in \LEPTO\ 6.5 \cite{Ingelman_LEPTO65}
(Fig.~\ref{DISmc}$b$.). 
\par 
In the case of BGF a color octet gluon is removed from the proton
leaving a quark and a $di - quark$ behind. The quark from BGF 
forms a color string with the $di - quark$ and the anti - quark from
BGF is connected via another string 
to the remnant valence quark (Fig.~\ref{DISmc}$c$.).
\par 
For QCD - Compton ($\gamma^* q_i \to q_f g_f$) the gluon $g_f$ acts as a kick 
in the color string drawn from the quark $q_f$ to the remnant $di - quark$
(Fig.~\ref{DISmc}$d$.).
If the quark $q_i$ actually was a sea quark, then the remnant has 
an additional anti-quark $\bar{q_i}$. This anti-quark 
$\bar{q_i}$ forms a color singlet state with the left over 
valence quark $q_v$ of the proton (Fig.~\ref{DISmc}$e$.). If the 
scattering occurred on a anti quark, then the strings are just reversed.
\par
In rapidity gap events the proton stays intact or becomes a low
mass diffractive state, here simply labeled with $p'$.
 Because of the emission of the
color neutral particle there is
no color connection between  the outgoing scattered proton $p'$
and the other particles.
%\clearpage
\begin{figure}[ht]
\begin{center}
\centerline{\includegraphics[bb=58 572 555 756,width=1\columnwidth]{difmc.epsi}}
\end{center}
\caption{Basic processes for inelastic diffractive 
lepton nucleon scattering.
Indicated are the color strings and the pomeron remnant.
$a$. shows the lowest order process for scattering quark.
$b$. shows the
$O(\alpha _{em}\alpha_s)$
for gamma gluon fusion (the crossed diagram is not shown).
 The pomeron remnant is a color octet gluon
 $c$. shows the
$O(\alpha _{em}\alpha_s)$
for QCD Compton (the crossed diagram is not shown) on a quark.
 The pomeron remnant is the same as in $a$.
\label{difmc}}
\end{figure}
When a quark (anti-quark) is removed from the  pomeron 
a anti-quark (quark) of the same flavor but with the corresponding anti-color
is left (Fig.~\ref{difmc}a). 
 When a gluon is removed from the
 pomeron $\PO$, a color octet remnant is left,  here
treated as a
single gluon. This pomeron remnant together with the $q\bar{q}$
of the hard interaction forms the color singlet state
(Fig.~\ref{difmc}b).
\par
Pion exchange is treated similarly to pomeron exchange with the 
corresponding modifications for the outgoing particle $p'$
which is a $p$, $n$ or $\Delta^{++}$  corresponding to 
$\pi^0$, $\pi^+$ or $\pi^-$ exchange. The treatment of valence and sea 
quarks follows the same principles as for inclusive $ep$ scattering described
above.
\par
The color strings are much more complicated for resolved photon processes 
because two remnants, one from the proton and one from the photon, have to
be considered. Resolved photon processes are similar to $p\bar{p}$ 
scattering and the relevant color connections are discussed in detail in
\cite{PYTHIA_ppbar_color}.
%Higher order gluon emission and subsequent gluon splitting
%into $q\bar{q}$ is included
%using the Color Dipole Model (ARIADNE) \cite{CDM}.
%%%%%%%%%%%%%%%%%%%%%%%%%%%%%%%%%%%%%%%
\subsection{Proton dissociation}
%%%%%%%%%%%%%%%%%%%%%%%%%%%%%%%%%%%%%%%
\subsubsection{Proton dissociation ala DIFFVM}
Dissociation of the proton 
according to the model in DIFFVM \cite{DIFFVM,DIFFVM2}
can be included for diffractive events 
via NFRAG=20. Additional parameters describe the dependence
$$
\frac{d \sigma }{dM_Y^2\,dt\,dx_{\PO}} \sim 
\frac{1}{M_Y^{2(1+\epsilon_Y)}}
\exp{\left(-B_{diss}|t|\right)} $$
with \verb+PEPS=+$\epsilon_Y$ and \verb+PRN2=+$B_{diss}$.
\subsubsection{Proton dissociation with a simple model}
Dissociation of the proton can be included for diffractive events 
via NFRAG=10.
 In this case the proton is split into a quark $q_p$ and di-quark $di-q_p$
system, whereas the pomeron is assumed to couple to a single quark 
$q_p$ only, 
and therefore the outgoing quark $q'_p$ carries 
 all of the momentum transfer $t$ giving it a transverse momentum. 
The quark to which the pomeron 
couples carries a momentum fraction $\chi$ of the protons initial
momentum. The momenta of the initial quark $q_p$ and the di-quark
$di-q_p$ are given in the following:
\begin{eqnarray}
q_p & \simeq & \chi p \\
di-q_p & \simeq & (1 - \chi) p \\
q'_p & = & q_p - p_{\PO} \simeq  \chi p - p_{\PO}
\end{eqnarray}
where  
$ \chi = \frac{q.q_p}{q.p}$ with
$q$ ($q_p$, $p$) being the photon (quark, proton) momentum.
 In addition the quark and
di-quark get primordial $p_{\perp}$ according to a Gaussian 
distribution. 
\par
The momentum fraction $\chi$ can be estimated within the 
resolved pomeron model:
\begin{equation}
\chi = \frac{\xpom}{\beta'}
\end{equation}
with $\beta' = \frac{q.p_{\PO}}{q.q_p}$ 
being the momentum fraction of the quark $q_p$ of the pomeron momentum,
and  $x_{\PO} = \frac{q.p_{\PO}}{q.p}$.
The value of $\xpom$ is already known from the interaction $\gamma \PO$
and $\beta'$ can be generated according to quark density of the pomeron.
\par
 For  $\beta'$ different probability functions
are available: 
\begin{eqnarray}
P(\beta')  = & 2 (1 - \beta') & (\verb+IREM=1+),\\ 
P(\beta')  =  & (a+1) (1 - \beta')^a &(\verb+IREM=2+),\\
P(\beta')  =  & \frac{N}{ \beta'\left(1 - \frac{1}{\beta'} - 
          \frac{c}{(1 - \beta')}\right) ^2}& 
                (\verb+IREM=3+)
\end{eqnarray}
 with $a$ chosen such that 
$<\beta'> = 1/(a + 2)$ 
and  $c$ determined by the ratio of masses of the
remnant quark and di-quark system. The option \verb+IREM=1+ 
corresponds to a hard quark density of the pomeron.
Actually all the parameterizations are taken 
from inclusive DIS scattering as implemented 
\LEPTO\ 6.1~\cite{MEPS}.
\par
The  mass distribution $M_R$ of the $p$ - dissociative
system follows in general a $1/M_R$ distribution for all the 
parameterizations of $P(\beta')$ available \cite{Jung_Eilat}.
\par
Even with proton dissociation switched on, a proton will emerge after
fragmentation when 
the momentum transfer is small 
and the mass of the $q$ - $ di-q$ system remains below
the threshold for multi-particle production.

%%%%%%%%%%%%%%%%%%%%%%%%%%%%%%%%%%%%%%%%%%
\subsection{QCD parton shower evolution}
%%%%%%%%%%%%%%%%%%%%%%%%%%%%%%%%%%%%%%%%%%

Higher order QCD effects are taken into account using the leading log
parton shower approach.
\subsubsection{Initial state radiation}
 Starting from the hard scattering process with
$x$ being the fractional momentum  of the incoming quark
 at a suitable scale $\mu ^2$,
 a backward evolution according to the DGLAP evolution
equations~\cite{DGLAPa,DGLAPb,DGLAPc,DGLAPd}
 will lead to larger initial values of 
$x_i$ and smaller $\mu ^2$.
Especially at high center of mass  energies 
initial state QCD radiation will
become important and has to be taken into account to model properly the
hadronic final state.
\par
The probabilities for a branching of a parton $a \to b c$
to happen are given by the DGLAP
evolution equations:
\begin{equation}
\frac{d\,f_a (x,t)}{d\,t} = \frac{\alpha_s (t)}{2 \pi}
        \sum_a \int_x ^1 \frac{d\,x'}{x'} f_a (x',t)
        P_{a \rightarrow b c} \left(\frac{x}{x'}\right)
\end{equation}
where $f_a (x',t)$ are the parton density functions, giving the
probability of finding a parton $a$ carrying the fraction $x'$ of the
momentum fraction $x$ probed at a  scale $t$. $P_{a \rightarrow b c} $
are the DGLAP splitting functions:
\begin{eqnarray}
  P_{q \rightarrow q g} (z) & = &\frac{4}{3} \frac{1 + z^2}{1-z} \\
  P_{g \rightarrow g g} (z) & = & 6 \frac{(1 - z(1-z))^2}{z(1-z)} \\
  P_{g \rightarrow q \bar{q}} (z) & = &\frac{1}{2}(z^2 +(1-z)^2) 
\end{eqnarray}
Soft gluon emission causes problems, since the splitting functions 
$P_{q \rightarrow q g},  P_{g \rightarrow g g}$ are divergent as
$z \rightarrow 1$. Practically 
in order to avoid divergences, a upper cutoff
$z_{max}$ is introduced and the remaining soft gluon emission is
treated as an effective shift in $z$ (for details
see \cite{PYTHIAPSa,PYTHIAPSb}). 
The actual value of $z_{max}=\frac{x+\epsilon}{x}$ plays a crucial role
in properties of the hadronic final state, such as the transverse 
energy flow in the proton direction. Using the standard value of \PYTHIA\ 
for $\epsilon$, which is $\sqrt{s} \cdot \epsilon = 2 $ GeV, the
transverse energy in the proton direction in DIS away from the current jet
falls well below the data \cite{H1_energyflow}. Decreasing $\epsilon$ 
increases the transverse energy flow, and the data are much better described.
The value $z_{max}$ chosen in \RAPGAP\ corresponds to the kinematic limit
of quark pair creation when the current masses of the quarks are taken into
account.
\par
 Starting from the hard interaction process at a
suitable scale $\mu ^2$ ($\mu ^2 = \hat{s},\PT ^2$)
the partons are evolved backward. This backward evolution is quite
complicated and is used in the Monte Carlo program for efficiency 
reasons when optimizing the generation of the kinematic variables for
the hard scattering process. In a forward evolution scheme the number of
degrees of freedom for the generation would vary event by event and no 
standard optimization procedure would be applicable.
\par
In $p\bar{p}$ collisions the ``$\hat{s}$ approach" is widely used
(\PYTHIA\ \cite{PYTHIAPSa,PYTHIAPSb}) 
whereas a different approach in lepto-production is
adapted in \LEPTO\ \cite{LEPTOPS,MEPS}.
The initiators of the parton shower cascade (partons inside the
proton for example) are treated collinear with the original particle
when effects from primordial $k_t$ are neglected and have negligible
mass ($m^2 \leq 1$ GeV). When a branching occurs, $p_3
\rightarrow p_1 + p_2$,
the daughter partons $p_1$ and $p_2$ will have transverse momenta
and virtualities greater than that of the initiator $p_3$. Thus
after the QCD cascade the partons going into the hard interaction
also have transverse momenta compared to the non parton shower case.
\par
In $\gamma p$ and $p\bar{p}$ collisions there is no problem
associated with this treatment since the kinematics of the hard
interaction usually cannot be determined by measuring the
scattered beam-particles either because the beam-particle is
totally absorbed (the photon in $\gamma p$) or because the
beam-particle is insufficiently measured by experiment. However in
lepto-production the kinematics are usually defined by the scattered
lepton ($y, Q^2$ for example). In that case there will be mismatch
between the generated $y,Q^2$ (before the parton cascade) and after
QCD radiation has been added mainly because of the additional
transverse momentum. For lepto-production a special approach has
been developed ``\LEPTO\ approach" \cite{LEPTOPS,MEPS}) 
in order to keep the scattered
lepton and the (virtual) photon unchanged even with initial state
parton shower. This approach is now consistently used in \RAPGAP\ within
the subroutine \verb+PYSSPA+ which was taken from \PYTHIA\ and 
\LEPTO\ ~\cite{Jetsetc,MEPS} with considerable modifications to be used
also for low $Q^2$ processes, photo-production  and diffractive scattering.
\par
In DGLAP based parton showers the emitted partons are strongly 
ordered in $p_T$, meaning that in a backward evolution the parton 
with the largest $p_T$ is generated first and closest in rapidity to the hard 
subprocess partons. The other partons generated must have smaller $p_T$.
\par
If the hard scattering was just the lowest order process 
($\gamma q \rightarrow q$) then the (only available and) largest scale is
$Q^2$ and the $p_T$ of the shower partons can go up to $\sqrt{Q^2}$. 
The situation is different when $O(\alpha_s)$ matrix element processes are
considered. The maximum scale could be $Q^2$, $\hat{s}$, $p_T^2$. Here we 
choose the maximum scale to be:
\begin{equation}
\mu^2 = \max(|-Q^2 - 2q.p_{f\;1},-Q^2 - 2q.p_{f\;2}|)
\end{equation}
This choice \cite{MEPS} is motivated from the propagator of the matrix 
elements and is of the order of $p_T^2$ as shown in eq.(~\ref{scale}).
\par
When $O(\alpha_s)$ processes are considered in addition to the 
leading order process ($\gamma q \rightarrow q$) then the maximum scale
for parton showers are calculated as follows: parton showers in the
leading order process can go up to $\mu^2 = 4 \cdot p_{T\;cut}^2$ instead of
$Q^2$ when no matrix elements are included (with $ p_{T\;cut}^2 =
\verb+PT2CUT+$ being the divergency cut off in the matrix elements). 
This choice can be understood since parton emissions with $p_T > p_{T\;cut}$
are already included in the matrix element simulation \cite{MEPS}.
Parton showers in the matrix element processes can go up to 
virtualities defined in eq.(\ref{scale}).
\par
It is obvious that the initial state parton shower 
approach can only be applied correctly
when parton densities are available as a function of $Q^2$. Moreover this
parton densities have to be obtained in a consistent way using the
DGLAP evolution equations and consistent definition of $\alpha_s$.
\par
Now also even for diffractive scattering QCD analysis of $F_2^{D(3)}$
 and parameterizations of the diffractive parton densities 
obtained with DGLAP are available and can be used in initial state 
parton showers.
%\par
% Since there is only small evolution from 
%$Q^2$ of a highly virtual photon up to the scale of the hard subprocess,
%initial state parton showers are not implemented for partons of the photon.
\par
Optionally QCD cascades can be simulated according to the 
color dipole model as implemented in
ARIADNE \cite{CDM}) (via \verb+IFPS=10+).
\subsubsection{Final state parton showers}
Final state parton showers are more easily simulated (\verb+PYSHOW+
of \PYTHIA) since here a forward evolution scheme is used and no parton 
densities enter the evolution. The maximum scale is either
$\min (4 p_{T\;cut}^2,W^2)$ for the leading order processes or $\hat{s}$ for
the $O(\alpha_s)$ matrix elements. Details can be found in 
\cite{Jetsetc}.
\section{Description of the program components}

\subsection{Subroutines and functions}

\begin{defl}{123456789012345}
\item[{\tt RGMAIN }]
                  main program
\item[{\tt GRAINI }]
                   initializes the program
\item[{\tt RAPGAP }]
    performs integration of the cross section. This routine has to be
            called before event generation can start.
\item[{\tt RAEND }]
            prints cross section and the number of events.
\item[{\tt EVENT }]
         performs the event generation and the proper mixing of 
         parton shower and matrix elements if selected via 
         {\tt IFULL=1}.
         Also the mixing of $standard$ $IS$, diffractive scattering
         and pion exchange is done, if selected via \verb+IDISDIF=1+.
\item[{\tt ANALYS }]
         user analysis subroutine.
\item[{\tt ALPHAS(RQ)}]
          give $\alpha_s (\mu)$ with $\mu = $RQ.
\item[{\tt PARTI }]
          initial particle and parton momenta are given.
\item[{\tt DFUN }]
          interface to FXN1
\item[{\tt FXN1 }]
          calls routines for selected processes:
          \verb+DIFFR1+, \verb+DIFFR2+, \verb+DIFFR3+, 
          \verb+DIFFR4+, \verb+DIS1+, \verb+DIS2+, \verb+DIS3+,
          \verb+DIS4+.
\item[{\tt CUTG(IPRO) }]
          cuts for process \verb+IPRO+=13 and \verb+IPRO+=15
          ($\gamma g$ fusion and QCD - Compton)  
          in integration and
          event generation.
\item[{\tt FRAG }]
          generates color string connection for hadronization.
\item[{\tt DIFFR1 }] for diffractive and pion exchange processes
  $ \gamma g_{\PO} \rightarrow q \bar{q}$.
          Calls kinematics and phase space routine \verb+PARTDF+
          and matrix element \verb+ELEQQL+ and \verb+ELEQQB+.
          Both for light and heavy quarks.
          Using Equivalent Photon approximation and $\gamma g \to
          q \bar{q}$ matrix element.
\item[{\tt DIFFR2 }] for diffractive and pion exchange processes
  $ e g_{\PO} \rightarrow e' q \bar{q}$.
          Calls kinematics and phase space routine \verb+PARTDF+
          and matrix element \verb+ELEQQF+. Both for light and heavy quarks.
          Using full matrix element for $e g \to e'
          q \bar{q}$.
\item[{\tt DIFFR3 }] for diffractive and pion exchange processes
        $e q (\bar{q}) \rightarrow e' q' (\bar{q}')$.
          Calls kinematics and phase space routine \verb+PARTDF+.
\item[{\tt DIFFR4 }] for diffractive and pion exchange processes
  $ e q_{\PO} \rightarrow e' q g$.
          Calls kinematics and phase space routine \verb+PARTDF+
          and matrix element \verb+ELEQCDC+. 
          Using full matrix element for $e q \to e'
          q g$.
\item[{\tt DIS1 }] for $standard$ inelastic scattering
  $ \gamma g_{p} \rightarrow q \bar{q}$.
          Calls kinematics and phase space routine \verb+PARTDI+
          and matrix element \verb+ELEQQL+ and \verb+ELEQQB+.
          Both for light and heavy quarks.
          Using Equivalent Photon approximation and $\gamma g \to
          q \bar{q}$ matrix element.
\item[{\tt DIS2 }] for $standard$ inelastic scattering
  $ e g_{p} \rightarrow e' q \bar{q}$.
          Calls kinematics and phase space routine \verb+PARTDI+
          and matrix element \verb+ELEQQF+. Both for light and heavy quarks.
          Using full matrix element for $e g \to e'
          q \bar{q}$.
\item[{\tt DIS3 }] for $standard$ inelastic scattering
        $e q (\bar{q}) \rightarrow e' q' (\bar{q}')$
          Calls kinematics and phase space routine \verb+PARTDI+.
\item[{\tt DIS4 }] for $standard$ inelastic scattering
  $ e q_{\PO} \rightarrow e' q g$.
          Calls kinematics and phase space routine \verb+PARTDI+
          and matrix element \verb+ELEQCDC+. 
          Using full matrix element for $e q \to e'
          q g$.
\item[{\tt ELEQQL }]
          matrix element for $\gamma g \rightarrow
          q \bar{q}$. $q$ stands for
          light quark.
\item[{\tt ELEQQB }]
          matrix element for $\gamma g \rightarrow
          Q \bar{Q}$ including masses. $Q$ stands for
          heavy quark.
\item[{\tt ELEQQF }]
          matrix element for $e g \rightarrow e'
          Q \bar{Q}$ including masses. $Q$ stands for light or
          heavy quark. Masses of light quarks $m_q = 10$ MeV.
\item[{\tt ELEQCDC }]
          matrix element for $e q \rightarrow e'
          q g$.  Masses of light quarks $m_q = 10$ MeV.
\item[{\tt ELERES }]
          matrix elements for resloved photon processes.
\item[{\tt ELEQQ }]
          matrix element for exlusive diffractive $q\bar{q}$ 
	    production~\cite{Bartels_jets,Bartels_asym}.
\item[{\tt ELEQQG }]
          matrix elements for diffractive $q\bar{q}g$ 
	    production~\cite{Bartels_qqg}.
\item[{\tt DOT(A,B) }]
         A.B four vector dot product
\item[{\tt DOT1( I,J)}]
          four vector dot product of vectors I and J in
          \verb+LUJETS+ common.
%\item[\verb+DVNOPT }]
%         changing options for \verb+DIVON+.
\item[{\tt RANUMS }]
         vector of random numbers used in event generation.
\item[{\tt PHASE }]
         phase space and generation for momenta of final
         partons in hard subprocess. $2 \rightarrow 2$ and
         $2 \rightarrow 3$ processes.
\item[{\tt PARTDF }]
         phase space and event record for diffractive and pion exchange
         processes.
\item[{\tt PARTDI }]
         phase space and event record for $standard$ inelastic scattering
         processes.
\item[{\tt PYSTFU(KF,X,SCALE,XPQ) }] parton density in particle \verb+KF+
            (\verb+KF+ = 2212 for the proton).
            \verb+XPQ+$ = xf_i(x,\mu^2)$ with \verb+X+$ = x$,
           \verb+SCALE+$ = \mu^2$.
            Copied from \LEPTO\ 6.1~\cite{MEPS} and updated to 
            include partons of the pomeron (pion) inside the proton.  
\item[{\tt RASTFU(KF,X,SCALE,XPQ) }] parton density in particle \verb+KF+
            (\verb+KF+ = 100 for the pomeron, \verb+KF+ = 211 for the pion).
            \verb+XPQ+$ = xf_i(x,\mu^2)$ with \verb+X+$ = x$,
             \verb+SCALE+$ = \mu^2$.
\item[{\tt RUNIGLU(KF,X,KT,P,XPQ) }]   
      to extract the unintegrated gluon density 
	$x {\cal A}(x,k_{t},\mu)$ for a proton with \verb+KF=2212+,
	as a function of $x=$\verb+X+, $k_{t}^2=$\verb+KT+ and $\mu=$\verb+P+.
      The gluon density is returned in \verb+XPQ(0)+, where \verb+XPQ+ is an 
      array with \verb+XPQ(-6:6)+.
\item[{\tt RAT2DI(KF,X,XMAX,TMIN,T,WTDIST) }]
            \verb+T+$=t$ 
            and \verb+X+$=r$ dependent probability distribution for
            radiating a parton \verb+KF+ from the proton.
            (\verb+KF+ = 100 for the pomeron, \verb+KF+ = 211 for the pion).
\item[{\tt PYREMN(IPU1,IPU2) }]  routine for remnant treatment.
            Copied from \LEPTO\ 6.1~\cite{MEPS} and updated for
            the use in resolved photon, diffractive and 
            pion exchange processes. 
\item[{\tt PRODIFF }]  routine for proton dissociation.
            The proton is treated as a quark di-quark system.
\item[{\tt PYSPLI(KF,KPA,KFSP,KFCH) }]  
            give the spectator \verb+KFSP+ and \verb+KFCH+
            partons when a parton
            \verb+KPA+ is removed from particle \verb+KF+.
            Copied from \LEPTO\ 6.1~\cite{MEPS} and updated for
            the use in resolved photon processes,
            diffraction and pion exchange. 
\item[{\tt LMEPS }]  routine for color flow in all
            processes and preparation for initial and final state
            parton showers.
            Copied from \LEPTO\ 6.1~\cite{MEPS} and updated for
            the use in low $Q^2$ processes, photo-production,
            resolved photon processes, diffraction and pion exchange. 
\item[{\tt PYSSPA(IPU1,IPU2) }]  routine for initial state radiation.
            Calls LSCALE.
            Copied from \LEPTO\ 6.1~\cite{MEPS} and updated for
            the use in low $Q^2$ processes, photo-production,
            resolved photon processes, diffraction and pion exchange. 
\item[{\tt LSCALE }]  gives the maximum virtuality
            to be used for initial state parton shower generation.
            Copied from \LEPTO\ 6.1~\cite{MEPS} and updated for
            the use for diffractive and pion exchange processes. 
\item[{\tt GADAP }]  Gaussian integration routine for 1-dim and 
            2-dim integration.
            Copied from \LEPTO\ 6.1~\cite{MEPS}.
\item[{\tt RALMKT }]  generate primordial $p_T$ according to the
                       Aligned Jet Model.
\item[{\tt RYSTGA }]  master routine for parton densities of the virtual 
                       photon. Calls \verb+GRSPAR+ and \verb+SASGAM+.
\item[{\tt GRSPAR }]  parameterization of Gluck, Reya, Stratman 
                       \cite{GRS}.
\item[{\tt SASGAM }]  parameterization of Schuler, Sj\"ostrand 
                       \cite{Sasgam}.
\item[{\tt F2DHW }] calculation of diffraction for \verb+NG=40+
                     \cite{Wusthoff}.
\item[{\tt F2BLW }] calculation of diffraction for \verb+NG=41+
                     \cite{Bartels_jets,Bartels_asym}.
\item[{\tt F2MD }] calculation of diffraction for \verb+NG=42+
                     \cite{Diehl1,Diehl2}.
\item[{\tt F2MCD }] calculation of diffraction for \verb+NG=45+
                     \cite{Buchmuller_Hebecker_Mcdermott}.       
\item[{\tt USDIFFR(BETA,SCALE,XPQ,X\_POM,T2) }]
          user supplied effective diffractive parton density 
          with the               
          fractional momentum \verb+BETA+$=\xi_i$ of the pomeron momentum 
          carried by the
          parton $i$, the scale \verb+SCALE+=$\mu$
          of the structure function \verb+XPQ(-6:6)+=$\xi q(\xi,\mu^2)$,
          \verb+X_POM+$=\xpom$ and \verb+T2+$=t$ (all variables in 
          \verb+SINGLE PRECISION+).
\end{defl}


\subsection{Parameter switches}
\begin{defl}{123456789012345}
\item[{\tt IINT:}] (D:=0) select integration procedure
\item[]  =0 BASES/SPRING Integration procedure
\item[]  =1 DIVON Integration procedure
\item[{\tt NCAL:}] (D:=10000) Nr of calls per iteration for bases
\item[{\tt ACC1:}] (D:=1)    relative precision (in \%) for grid optimisation
\item[{\tt ACC2:}] (D:=0.5)  relative precision (in \%) for integration
\end{defl}

\subsubsection{Parameters for kinematics in lepton-hadron collisions}
\begin{defl}{123456789012345}
\item[{\tt PLEPIN:}]\index{PLEPIN}
                        momentum $p$ [GeV/$c$]
                        of incoming electron (D=$-30$) (\verb+/INPU/+)
\item[{\tt PIN:}]\index{PIN}
                        momentum $p$ [GeV/$c$]
                        of incoming proton (D=$820$)(\verb+/INPU/+)
\item[{\tt QMI:}]\index{QMI} 
       (D: = 5.0) (\verb+/DIFFR/+)Minimum $Q^2$ to be generated 
\item[{\tt QMA:}]\index{QMA}
 (D: = $10^8$) (\verb+/DIFFR/+)Maximum $Q^2$ to be generated
\item[{\tt YMI:}]\index{YMI} 
(D: = 0.0)(\verb+/DIFFR/+) Minimum $y$ to be generated
\item[{\tt YMA:}]\index{YMA}  
(D: = 1.0) (\verb+/DIFFR/+)Maximum $y$ to be generated
\item[{\tt THEMA,THEMI}]\index{THEMA,THEMI}
  (D: \verb+THEMA+ = 180., \verb+THEMI+ = 0)
                    maximum and minimum scattering angle $\theta$ of the 
                    electron (\verb+/ELECT/+).
\item[{\tt NFLAV}] \index{NFLAV}
(D: = 5) number of active flavors, can be set by user.
                    (\verb+/LUCO/+)
\item[{\tt NFLQCDC}]\index{NFLQCDC} 
(D: = 3) number of flavors allowed for 
QCD Compton processes, can be set by user (\verb+/LUCO/+).


\end{defl}
\subsubsection{Parameters for kinematics in hadron-hadron collisions}
\begin{defl}{123456789012345}
\item[{\tt PLEPIN:}]\index{PLEPIN}
                        momentum $p$ [GeV/$c$]
                        of incoming hadron 1  (\verb+/INPU/+)
\item[{\tt PIN:}]\index{PIN}
                        momentum $p$ [GeV/$c$]
                        of incoming hadron 2 (\verb+/INPU/+)
\item[{\tt NFLAV}] \index{NFLAV}
(D: = 5) number of active flavors, can be set by user.
                    (\verb+/LUCO/+)
\item[{\tt NFLQCDC}]\index{NFLQCDC} 
(D: = 3) number of flavors allowed for 
QCD Compton processes, can be set by user (\verb+/LUCO/+).


\end{defl}


\subsubsection{Parameters for hard subprocess selection}
\begin{defl}{123456789012345}
\item[{\tt IPRO:}]\index{IPRO} (D: = 12) (\verb+/RAPA/+)
                    select hard subprocess to be generated. The 
                    selection of pomeron, pion or $standard$ inelastic 
                    scattering is done via 
                    \verb+IDIR+, \verb+IDISDIF+, \verb+NG+, \verb+NPOM+
                    described below. 
\item[{\it         }]
                 =10:   $\gamma g \rightarrow q \bar{q}$
                        using EPA
\item[{\it         }]
                 =11:   $\gamma g \rightarrow c \bar{c}$
                        using EPA
\item[{\it         }]
                 =12:   $e q \rightarrow e' q'$
\item[{\it         }]
                 =13:   $e g \rightarrow e' q \bar{q}$
                        using full Matrix Element
\item[{\it         }]
                 =14:   $e  g \rightarrow e' c \bar{c}$
                        using full Matrix Element
\item[{\it         }]
                 =15:   $e  q \rightarrow e' q g$
                        using full Matrix Element
\item[{\it         }]
                 =16:   $e  q \rightarrow e' q \gamma$
                        using EPA with $\gamma q \to \gamma q$.
\item[{\it         }]
                 =18:   resolved photon processes, selected via 
                        using full Matrix Element
\item[{\it         }]
                 =20:   $\gamma^* p \to q \bar{q} g p$~\cite{Bartels_qqg}.
\item[{\it         }]
                 =21:   $\gamma^* p \to q \bar{q} p$
		            \cite{Bartels_jets,Bartels_asym}.
		            Exclusive diffractive
		            dijet production, the same process as \verb+IPRO=12+
				together with \verb+NG=41+ and \verb+NPOM=41+, but
				technically calculated differently and more efficient. 
				
\item[{\it         }]
                 =30: saturation model for   $\gamma^* p \to q \bar{q} p$
		     $\gamma^* p \to q \bar{q} g p$~\cite{satrap}.
\item[{\it         }]
               =1200:   use HERACLES \cite{HERACLESa,HERACLESb} 
                        (optional for 
                        simulation of QED radiation) for 
                        $e q \rightarrow e' q'$
\item[{\it         }]
               =1400:   use HERACLES \cite{HERACLESa,HERACLESb} 
                        (optional for 
                        simulation of QED radiation) for heavy quark production 
				via boson gluon fusion: 
                        $e g \rightarrow e' Q \bar{Q}$ 

\item[{\tt Q2SUPP}]\index{Q2SUPP} (D=3.37)  exponential low $Q^2$ suppression 
                      of the parton
                      densities to be used with HERACLES:
                      $1-\exp{(-\verb+Q2SUPP+\cdot Q^2}$, can be
                     changed by user (\verb+/LOWQ2S/+).

\item[{\tt IBS}]\index{IBS}  
(D: = 0) select type order $alpha_s$ corrections. 
\item[]                    = 0 using $p_t$ cutoff scheme
\item[]                    = 1 using Collins scheme for NLO 
             \cite{Collins-collfac,Collins-collfac2} 
         .
                 
\item[{\tt IDIR}]\index{IDIR}  
(D: = 0) select type of events to be generated.
                 (\verb+/DISDIF/+) 
\item[]                    = 1 $standard$ inelastic scattering 
\item[]                    = 0 diffractive and pion exchange processes
\item[{\tt IDISDIF}]\index{IDISDIF}
  (D: = 0) choose mixing of $standard$ inelastic
                    scattering, diffractive and pion exchange processes
                    according to cross section(\verb+/DISDIF/+).
\item[]                    = 0 generates only the processes selected
                    by \verb+IDIR+. 
                   If \verb+IDIR+ = 0 then pomeron or pion exchange 
                    is selected via \verb+NG+ and \verb+NPOM+.  
\item[]                    = 1 mixing of $standard$ inelastic and diffractive 
                    processes.  
\item[]                    = 2 mixing of $standard$ inelastic, diffractive and
                    pion exchange processes.
\item[{\tt IFULL}]\index{IFULL}
                   (D: = 1) switch to select lowest order process
                   (\verb+IFULL+ = 0) or quark parton model with 
                   $O(\alpha_s)$  matrix elements (\verb+IFULL+ = 1)
			 (\verb+/OALPINI/+).
\item[{\tt IQCDGRID}]\index{IQCDGRID}
                   (D: = 1) switch to select 
                   generation of $O(\alpha_s)$ processes in a grid
                   (\verb+/OALPINI/+). 


\item[{\tt INTER:}]\index{INTER} (D: =0) interaction type(\verb+/INPU/+)
\item[             ] = 0 neutral current photon interaction
\item[             ] =2 charged current interaction
\item[{\tt SIN2W}]\index{SIN2W} (D:=0.23) for electroweak processes 
                   $\sin ^2_W$ (\verb+/EWEAK+). 
\item[{\tt XMW2}]\index{XMW2}
 (D:= $80 \cdot 80$ GeV$^2$) for electroweak processes  $m_W^2$ 
                   (\verb+/EWEAK+).


\item[{\tt ISEMIH:}]\index{ISEMIH} not used at present(\verb+/INPU/+)

\item[{\tt PT2CUT(IPRO):}]\index{PT2CUT}
 (D=5.0) minimum $\hat{p}^2 _{\perp}$ for
                            process \verb+IPRO+ (\verb+/PTCUT/+).
                            Must be used for generation of light quarks
                            in processes \verb+IPRO=10,13,15,18+.
                            

\end{defl}
\subsubsection{Parameters for parton shower and fragmentation}
\begin{defl}{123456789012345}
\item[{\tt NFRAG:}]\index{NFRAG} (D: = 1)
                        switch for fragmentation(\verb+/INPU/+)
\item[] = 0 off
\item[] = 1 on 
\item[] = 10 proton dissociation in simple model
\item[] = 20 proton dissociation ala DIFFVM
\item[{\tt IFPS:}]\index{IFPS} (D: = 3)
                  switch  parton shower(\verb+/INPU/+)
\item[] = 0 off
\item[] = 1 initial state
\item[] = 2 final state
\item[] = 3 initial and final state  
\item[] = 10 gluon radiation according to ARIADNE.

\item[{\tt ILHA:}]\index{ILHA} (D: = 0) Les Houches accord interface     
\item[] = 0 off
\item[] = 1  use Les Houches accord interface to write parton level event on
file \verb+rapgap.gen+


\item[{\tt IORD:}]\index{IORD} (D: = 1) ordering for initial state P.S.
\item[] =0 no ordring
\item[] =1 $Q^2$ values at branches are strictly
             ordered, increasing towards the hard
             scattering strict ordered in $Q^2$
\item[]=2 $Q^2$ and opening angles of emitted
          (on shell or time like) partons
           are both strictly ordered, increasing
          towards the hard interaction as 1
          but also strict ordered in angle
\item[{\tt IALP:}]\index{IALP} (D: = 1)  
\item[] =1 $\alpha_s$ first order with scale $Q^2$
\item[]=2 $\alpha_s$ first order with scale
           $k_t^2=(1-z) \cdot Q^2$
\item[{\tt ITIM:}]\index{ITIM} (D: =1)
\item[] =0 no shower of time like partons
\item[] =1 time like partons may shower
\item[{\tt ISOG:}]\index{ISOG} (D: =1) treatment of soft gluons
\item[] =0 soft gluons are entirely neglected
\item[] =1 soft gluons are resummed and included
           together with the hard radiation
           as an effective $z$ shift
 
\item[{\tt KT1:}]\index{KT1} 
(D:=0.7)  width of a gaussion for intrinsic $k_t$ for photon
 
\item[{\tt KT2:}]\index{KT2} 
 (D:=0.44) width of a gaussion for intrinsic $k_t$ for proton

\end{defl}



\subsubsection{Parameters for resolved photon processes}
\begin{defl}{123456789012345}
\item[]  process and color configuration for 
                          resolved photon processes (\verb+ /COLCON/+)

\item[IRPA ]\index{IRPA} =1(0) 
(D:=1) process $gg \rightarrow q \bar{q}$ switched on(off)
\item[IRPB ]\index{IRPB} =1(0)
 (D:=1) process $ g g \rightarrow gg $  switched on(off)
\item[IRPC ]\index{IRPC} =1(0) 
(D:=1) process $ q g \rightarrow q g $ switched on(off)
\item[IRPD ]\index{IRPD} =1(0)
 (D:=1) process $ q \bar{q} \rightarrow g g $  switched on(off)
\item[IRPE ]\index{IRPE} =1(0)
 (D:=1) process $ q \bar{q} \rightarrow q \bar{q} $  switched on(off)
\item[IRPF ]\index{IRPF} =1(0)
 (D:=1) process $ q q \rightarrow q q $  switched on(off)
\item[IRPG ]\index{IRPG} =1(0) 
(D:=0) process $ q q \rightarrow q q $ (BFKL)  switched on(off)
\item[IRPH ]\index{IRPH} =1(0) 
(D:=1) process $ q g \rightarrow q \gamma $ switched on(off)
\item[IRPI ]\index{IRPI} =1(0)
 (D:=1) process $ q \bar{q} \rightarrow g \gamma $  switched on(off)

 
\item[{\tt SCALQ2}]\index{SCALQ2} (D:=1.) specifies the cut on 
$\frac{\mu^2}{Q^2}=\verb+SCALQ2+$ for resolved processes in DIS, can be
                      changed by user (\verb+/RESGAM+).


\item[{\tt OMEG2:}]\index{OMEG2}
                   (D:=0.01)
                   $\omega=\verb+OMEG2+$ suppression factor for 
                   virtual resolved photons in the Dress - Godbole 
                   model~\cite{Drees_Godbole}.



\end{defl}

\subsubsection{Parameters for structure functions $\alpha_s$ and scales}
\begin{defl}{123456789012345}
\item[{\tt IRUNAEM:}]\index{IRUNAEM} (D: = 0)
                   select running of $\alpha _{em}(Q^2)$
\item[]
                        =0:  no running of $\alpha _{em}(Q^2)$
\item[]
                        =1:  running of $\alpha _{em}(Q^2)$


\item[{\tt IRUNA:}]\index{IRUNA} (D: = 1)
                        switch for running $\alpha _s$
\item[]
                        =0:  fixed $\alpha_s=0.3$ 
\item[]
                        =1: running $\alpha _s(\mu^2)$
\item[{\tt IQ2:}]\index{IQ2} (D: = 5)
                   select scale $\mu^2$ for $\alpha _s(\mu^2)$
\item[]
                        =1:  $\mu^2 = 4 \cdot m_{q} ^2$
                             (use only for heavy quarks!)
\item[]
                        =2:  $\mu^2 = \hat{s} $
                             (use only for heavy quarks!)
\item[]
                        =3:  $\mu^2 = 4 \cdot m^2 + p_{\perp} ^2$
                             
\item[]
                        =4:  $\mu^2 = Q^2$
                        
\item[]
                        =5:  $\mu^2 = Q^2 + p_{\perp} ^2$

\item[{\tt SCALFA}]\index{SCALFA} (D=1) (\verb+/SCALF/+)
                     factor which the scale $\mu^2$ used 
                     in $\alpha_s$ and
                     structure function evaluation 
                     is multiplied with. For example 
			   \verb+SCALFA=4+ and \verb+IQ2=3+ means:
			   $\mu^2= 4\cdot m^2 + 4 \cdot p_{\perp} ^2$, whereas for 
			   \verb+IQ2=5+ the meaning is 
			   $\mu^2= Q^2 + 4 \cdot p_{\perp} ^2$.

\item[{\tt MSTP(51):}]\index{MSTP(51)} \verb+MSTP(51)+, \verb+MSTP(51)+
                         \verb+MSTP(56)+ is used to select
                     structure function  parameterizations 
				 (\verb+/PYPARS/+). With \verb+MSTP(51)+ $<10$ 
                        the parameterization of $p$ structure function
                         from PYSTFU are used:\\
                         = 0: Simple scaling Function\\
                         = 1: EHLQ set 1 \\
                         = 2: EHLQ set 2 \\
                         = 3: Duke-Owens set 1 \\
                         = 4: Duke-Owens set 2 \\
                         = 5: Morfin-Tung set 1 (S1) \\
                         = 6: Morfin-Tung set 2 (B1) \\
                         = 7: Morfin-Tung set 3 (B2) \\
                         = 8: Morfin-Tung set 4 (E1) \\
                         = 9: Gluck-Reya-Vogt LO set \\
                         = 10: Gluck-Reya-Vogt HO set \\
                         \verb+MSTP(51)+ $>10$
                         parameterization from LHAPDFLIB 
                         \cite{\LHAPDFLIB}
                          is used \\
                             example: 10150
                             for CTEQ6L NLO proton parton density.  \\
\item[{\tt MSTP(52):}]\index{MSTP(52)} \verb+MSTP(52)+ $<10$ 
                        parameterization of $\pi$ structure function
                         from \verb+PYSTFU+ (Owens set) is used\\
                         \verb+MSTP(52)+ $>10$ 
                         parameterization from LHAPDFLIB is used\\
                            example: 211 for 
                             for GRV LO \cite{GRVa,GRVb} 
                             pion parton density.  
\item[{\tt MSTP(56):}]\index{MSTP(56)} \verb+MSTP(56)+ $<10$ 
                         inbuild parton densities are used:\\
                         = 1: GRS \cite{GRS}\\
				 = 2: SaS \cite{Sasgam}\\				 
                        \verb+MSTP(56)+ $>10$ 
				parameterization from LHAPDF is used\\
                            example: Nset= 391
                             for SaS \cite{Sasgam} 
                             virtual photon parton density together with
				     the Drees-Godbole $Q^2$ \cite{Drees_Godbole}
				     suppression factor.  

\end{defl}

\subsubsection{Parameters for diffraction}

\begin{defl}{123456789012345}
\item[{\tt NG:}]\index{NG} (D: = -14) select pomeron structure 
                        function
                        $xf(x)$ (\verb+/DIFFR/+). 
\item[]
                        = 0:  $xf_0(x)= 6x(1-x)$ for gluons. For quarks 
                        $xq(x) =\frac{1}{4} xf_0(x)$
\item[]
                        = n:  $xf_n(x)= (n+1)(1-x)^n$
                              for $1 \leq n \leq 5$ for gluons. For quarks 
                        $xq(x) =\frac{1}{4} xf_n(x)$
\item[]                 = 10: $xf(x) = (0.18 + 5.46x)(1-x)$ only gluons.
\item[]
                        = 11:  $xq(x) = \frac{1}{3}C\pi x(1-x)$
                             Donnachie Landshoff quark density in
                             pomeron.
\item[]
                        = 12: Kniehl, Kohrs, Kramer parton density
                             \cite{Kohrsa,Kohrsb} for
                             pomeron including direct coupling. 
\item[]
                        = 20: parton density for $\pi ^{\pm}$.
                              If \verb+MSTP(52)+ $ < 10 $ 
                               Owens set from PYSTFU is used.
                              IF \verb+MSTP(52)+ $ > 10 $ 
                              parton density for pion used
                              from LHAPDFLIB 
                             \cite{\LHAPDFLIB} 
\item[]
                        = 21: parton density for $\pi ^0$.
                              If \verb+MSTP(52)+ $ < 10 $ 
                              Owens set from PYSTFU is used.
                              IF \verb+MSTP(52)+ $ > 10 $ 
                              parton density for pion used
                              from LHAPDFLIB 
\item[] = 30: Nikolaev, Zakharov model \cite{NikZak}.
\item[] = 40: W\"usthoff model \cite{Wusthoff}.
\item[] = 41: Bartels, Lotter, W\"usthoff model 
                              \cite{Bartels_jets,Bartels_asym}.
\item[] = 42: Diehl model \cite{Diehl1,Diehl2}.
\item[] = 45: Buchm\"uller, Hebecker, McDermott model
                              \cite{Buchmuller_Hebecker_Mcdermott}.
\item[] = -10: H1 fit 1 (quarks only) (NLO)\cite{H1_F2D3_97}
\item[] = -11: H1 fit 2 (quarks and gluons) (NLO)\cite{H1_F2D3_97} 
\item[] = -12: H1 fit 3 (quarks and gluons peaked at $z \to 1$)
                        (NLO)\cite{H1_F2D3_97} 
\item[] = -13: H1 fit 1 (quarks only)(LO) \cite{H1_F2D3_97}
\item[] = -14: H1 fit 2 (quarks and gluons) (LO)\cite{H1_F2D3_97} 
\item[] = -15: H1 fit 3 (quarks and gluons peaked at $z \to 1$) 
                        (LO)\cite{H1_F2D3_97} 
\item[] = -30: H1 set A (NLO) \cite{H1_F2D3_06}
\item[] = -31: H1 set B (NLO) \cite{H1_F2D3_06} 

\item[] $<0$: user supplied structure function via 
                           subroutine \\
                           \verb+USDIFFR+.
                           The parton density must be put in the
                           array \verb+XPQ(-6:6)+
                           with the gluon at position $0$,
                           $u$, $d$, $s$, $c$, $b$, $t$ quarks at
                           positions $1,2,3,4,5,6$ and the anti-quarks at 
                           $-1,-2,-3,-4,-5,-6$.  
\item[{\tt NPOM:}]\index{NPOM} (D: = -10) select pomeron distribution
                    $f_{p/\PO}$ (\verb+/DIFFR/+)
\item[] = 0: pomeron distribution $f^S_{p/\PO}$
\item[] = 1: pomeron distribution $f^{IS}_{p/\PO}$
\item[] = 2: pomeron distribution $f^{DL}_{p/\PO}$
\item[] = 20: $\pi^-$  $t$ - distribution \cite{Holtmann}
\item[] = 21: $\pi^0$ $t$ - distribution \cite{Holtmann}
\item[] = 22: $\pi^+$  $t$ - distribution \cite{Holtmann}
\item[] = 30: Nikolaev, Zakharov model \cite{NikZak}.
\item[] = 40: W\"usthoff model \cite{Wusthoff}.
\item[] = 41: Bartels, Lotter, W\"usthoff model 
                              \cite{Bartels_jets,Bartels_asym}.
\item[] = 42: Diehl model \cite{Diehl1,Diehl2}.
\item[] = 45: Buchm\"uller, Hebecker, McDermott model
                              \cite{Buchmuller_Hebecker_Mcdermott}.
\item[] = -10: H1 fit pomeron only \cite{H1_F2D3_97} 
\item[] = -11: H1 fit meson only \cite{H1_F2D3_97} 
\item[] = -12: H1 fit pomeron and meson \cite{H1_F2D3_97} 
\item[] = -30: H1 fit pomeron only \cite{H1_F2D3_06} 
\item[] = -31: H1 fit meson only \cite{H1_F2D3_06} 
\item[] = -32: H1 fit pomeron and meson \cite{H1_F2D3_06} 
\item[] $<0$: user supplied pomeron distribution via 
                           subroutine \verb+USDIFFR+.

\item[ ] Parameters for the diffractive
                           gluon density in the semi-classical approach of
                            Buchm\"uller, Hebecker, McDermott
                              \cite{Buchmuller_Hebecker_Mcdermott}.
\item[{\tt C1:}]\index{C1} (D: = 1) $C_1=\verb+C1+$ (\verb+/BUCHMUE/+).
\item[{\tt Cg:}]\index{Cg} (D: = 1) $C_g=\verb+Cg+$ (\verb+/BUCHMUE/+).

\item[{\tt Iqqg}] \index{Iqqg} allows different levels of appriximation in the
matrix element calcualtion of \verb+IPRO=20+. Iqqg=0 is the full matrix element,
Iqqg=1,2 approximates the matrix element as described in \cite{Bartels_qqga}

\item[{\tt IGLU:}] \index{IGLU} (D: = 1)
			select unintegrated gluon density{\tt /GLUON/}).
\item[]
                        =1: CCFM set A0 \cite{jung-dis04}
\item[]
                        =2:  derivative of GRV~\cite{GRV95}
				$\frac{d xg(x,Q^2)}{dQ^2}$.
\item[]
                        =3:  approach of Bl\"umlein~\cite{Bluemlein}.
\item[]
                        =4:  KMS~\cite{martin_stasto} (\verb+kms.dat+).
\item[]
                        =5:  saturation model~\cite{wuesthoff_golec-biernat}.
\item[]
                        =6:  KMR~\cite{martin_kimber} (\verb+kmr.dat+).




\item[{\tt IREM:}]\index{IREM} (D: = 1) 
                     select momentum distribution of proton
                             dissociation (\verb+/PREMNANT/+).
\item[     ] = 1: $P(\beta')  =  2 (1 - \beta') $
\item[     ] = 2: $P(\beta')  =   (a+1) (1 - \beta')^a $
\item[     ] = 3: $P(\beta')  =  
           \frac{N}{ \beta'\left(1 - \frac{1}{\beta'} - 
           \frac{c}{(1 - \beta')}\right) ^2}$

\item[{\tt IVM:}]\index{IVM} (D: = 0) select exclusive vector
meson production in diffractive scattering (\verb+/VMESON+).
\item[] = 0: no special selection done.
\item[] $>1$ and $< 443$ :production of light vector mesons
\item[] $= 443$ :production of $J/\psi$ mesons.
\item[] $= 553$ :production of $\Upsilon$ mesons.


\item[{\tt IALMKT:}]\index{IALMKT} (D: =0)(\verb+/INPU/+)
                   include primordial $k_t$ for diffractive processes
                   according to the Aligned Jet Model:   
                   $e^{-5.5 k_t}$ for \verb+IALMKT+=1 (\verb+/INPU/+).

\item[{\tt T2MAX:}]\index{T2MAX} (D: = 5 ) 
                        maximum $t$  [GeV$^2$/$c^2$] (\verb+/DIFFR/+).

\item[{\tt XF:}]\index{XF}  (D: = 0.9)
                        minimum \verb+XF+ $x_\pom=1-x_f=1-\frac{E_p'}{E_{p}}$ 
				(\verb+/DIFFR/+).
\item[Parameters for proton dissociation:] 	
\item[{\tt PEPS:}]\index{PEPS} (D:=0.0) $\epsilon_Y$ for $M_Y^2$ spectrum
\item[{\tt PRN2:}]\index{PRN2} (D:=4.0) $B_{diss}$ for $\exp(-B_{diss} |t|)$
\item[Parameters for pomeron flux (only for {\tt NPOM=0,1,2)}:] 				
\item[{\tt ALPHP:}]\index{ALPHP}
 (D: = 0.25) $\alpha_{\PO}$ [GeV$^{-2}$] (\verb+/DIFFR/+).
\item[{\tt RN2:}]\index{RN2} (D: = 4.7)
                        \verb+RN2+$=b_0$ as defined above (\verb+/DIFFR/+).
\item[{\tt EPSP:}]\index{EPSP}
 (D: = 0.085) \verb+EPSP+$=\epsilon$ (\verb+/DIFFR/+). 


                       
\end{defl}

\subsubsection{Accessing information}
\begin{defl}{123456789012345}
\item[ ]

\item[{\tt AVGI}]\index{AVGI} integrated cross section (\verb+/EFFIC/+).
\item[{\tt SD}]\index{SD} standard deviation of integrated cross section
                 (\verb+/EFFIC/+).

\item[ ]
\item[{\tt SSS}]\index{SSS} 
total center of mass energy  $s$ (\verb+/PARTON/+)

\item[{\tt PBEAM}]\index{PBEAM} energy momentum vector of beam particles
  (\verb+/BEAM/+)
\item[{\tt KBEAM}]\index{KBEAM} flavor code of beam particles
 (\verb+/BEAM/+)


\item[{\tt Q2}]\index{Q2} in lepto-production: actual $Q^2$ of
                         $\gamma$ (\verb+/PARAE/+).
\item[{\tt YY}]\index{YY}
                        energy fraction lost by incident electron
				(\verb+/RAPGKI/+)
\item[{\tt XEL}]\index{XEL}
                        energy fraction of parton on electron side
				(\verb+/RAPGKI/+)
\item[{\tt XPR}]\index{XPR}
                        energy fraction of parton on proton side
				(\verb+/RAPGKI/+)
\item[{\tt YMAX,YMIN}]\index{YMAX,YMIN} actual upper and lower limits for $y$ 
                          (\verb+/PARAT/+).
\item[{\tt Q2MAX,Q2MIN}]\index{Q2MAX,Q2MIN} actual upper and
                        lower limits for $Q^2$ of $\gamma$ (\verb+/PARAT/+).
\item[{\tt XMAX,XMIN}]\index{XMAX,XMIN} 
upper and lower limits for $x$ (\verb+/PARAT/+).


\item[{\tt PT2H}]\index{PT2H}
                        $\hat{p}_{\perp} ^2$
     [GeV$^2$/$c^2$]
                      of parton in hard subprocess $cm$
			     (\verb+/RAPGKI/+)
system
\item[{\tt SHH}]\index{SHH}
                        invariant mass $\hat{s}$ [GeV$^2$]
                     of hard subprocess (\verb+/RAPGKI/+)
\item[{\tt T2GKI}]\index{T2GKI}
                        for diffractive processes 
                        \verb+T2GKI+ = $t$ [GeV$^2$] (\verb+/RAPGKI/+)
\item[{\tt XFGKI}]\index{XFGKI}
                        for diffractive processes 
                        \verb+XFGKI+ = $x_{\PO}$ (\verb+/RAPGKI/+)
                                                

\item[ ] $O(\alpha_s)$ matrix element information
                         (\verb+/MEINFO/+)
\item[{\tt AM(18)}]\index{AM}
 vector of masses of final state particles of hard
                         interaction.
\item[{\tt SHAT}]\index{SHAT} $\hat{s}$ of hard subprocess (\verb+/PARAT/+)
\item[{\tt ZQGKI}]\index{ZQGKI} 
                $z=\frac{p_i.p_f}{p_i.q}=\verb+ZQGKI+$ (\verb+/MEINFO/+)
\item[{\tt XPGKI}]\index{XPGKI}
              $x_p=\frac{Q^2}{2p_i.q}=\verb+XPGKI+$ (\verb+/MEINFO/+)
\item[{\tt PHIGKI}]\index{PHIGKI}  $\phi=\verb+PHIGKI+$ azimuthal
                     angle (\verb+/MEINFO/+)

\item[{\tt NIA1,NIA2}]\index{NIA1,NIA2}
 position of partons in hard interaction in
                         \verb+LUJETS+ event record (\verb+/HARD/+)
\item[{\tt NF1,NF2}]\index{NF1,NF2} first and last position final
                         partons/particles of
                         hard interaction in \verb+LUJETS+ (\verb+/HARD/+)
\item[{\tt NFT}]\index{NFT} total number of final particles; for
                         $2 \rightarrow 2$ process \verb+NFT+ =2
                        

\item[{\tt Q2Q}]\index{Q2Q} hard scattering scale $\mu ^2$ used in
                         $\alpha_s$ and structure functions (\verb+/PARAE/+).
\item[{\tt ALPHS}]\index{ALPHS} actual $\alpha_s$ (\verb+/PARAM/+).
\item[{\tt PI}]\index{PI} $\pi$ (\verb+/PARAM/+).
\item[{\tt ALPH}]\index{ALPH} $\alpha_{em}$ (\verb+/PARAM/+).
\item[{\tt NIN}]\index{NIN} 
number of trials for event generation (\verb+/EFFIC/+).
\item[{\tt NOUT}]\index{NOUT} number of successful generated events (\verb+/EFFIC/+).
\item[{\tt SCAL1,SCAL2}]\index{SCAL1,SCAL2}
 scale for structure function on beam 1 and 2 
                               respectively (\verb+/STRU/+).
\item[{\tt XDP1,XPD2}]\index{XDP1,XPD2} 
value of parton density on beam 1 and 2 
                               respectively (\verb+/STRU/+).


\end{defl}






\subsection{List of COMMON blocks}

\verb+COMMON/BEAM/PBEAM(2,5),KBEAM(2,5)+ \\
\verb+COMMON/BUCHMUE/C1,Cg+\\
\verb+COMMON/COLCON/ICOLORA,IRESPRO,IRPA,IRPB,IRPC,IRPD,IRPE,IRPF,IRPG+\\
\verb+COMMON/DIFFR/T2MAX,XF,ALPHP,RN2,EPSP,QMI,YMI,QMA,YMA,NG,NPOM+\\
\verb+COMMON/DISDIF/IDIR,IDISDIF+\\
\verb+COMMON/EFFIC/AVGI,SD,NIN,NOUT+\\
\verb+COMMON/ELECT/THEMA,THEMI+\\
\verb+COMMON/EWEAK/SIN2W,XMW2+\\
\verb+COMMON/F2INT/F2DIS,F2DIF,F2PI+
\verb+COMMON/HARD/NIA1,NIR1,NIA2,NIR2,NF1,NF2,NFT+\\
\verb+COMMON/INPU/PLEPIN,PPIN,NFRAG,ILEPTO,IFPS,IHF,IALMKT,INTER,ISEMIH+\\
\verb+COMMON/LOWQ2S/Q2SUPP+\\
\verb+COMMON/LUCO/KE,KP,KEB,KPH,KGL,KPA,NFLAV,NFLQCDC+\\
\verb+COMMON/MEINFO/ZQGKI,XPGKI,PHIGKI+\\
\verb+COMMON/OALPINI/IFULL,IQCDGRID+\\
\verb+COMMON/PARAE/Q2,Q2Q,PCM(4,18)+\\
\verb+COMMON/PARAM/ALPHS,PI,ALPH,IWEI+\\
\verb+COMMON/PARAT/AM(18),SHAT,YMAX,YMIN,Q2MAX,Q2MIN,XMAX,XMIN+\\
\verb+COMMON/PARTON/SSS,CM(4),DBCMS(4)+ \\
\verb+COMMON/PREMNANT/IREM+\\
\verb+COMMON/PTCUT/PT2CUT(100)+\\
\verb+COMMON/PYPARS/MSTP(200),PARP(200)+ \\
\verb+PARAMETER (NY=40,NQ=20)+ \\
     \verb+COMMON/QCDGRI/QY(NY),QQ(NQ),+ \\
     \verb+& QPMDF(NY,NQ),QQBDF(NY,NQ),QQBHDF(NY,NQ),QCDCDF(NY,NQ),+\\
     \verb+& QPMPI(NY,NQ),QQBPI(NY,NQ),QQBHPI(NY,NQ),QCDCPI(NY,NQ),+\\
     \verb+& QPM(NY,NQ),QQB(NY,NQ),QQBH(NY,NQ),QCDC(NY,NQ)+  \\
\verb+PARAMETER (NBQ2=20,NBX=20)+\\
     \verb+COMMON/RAHER/IHERPYS,XPQDIF(-6:6,NBX,NBQ2),+%
     \verb+XPQPI(-6:6,NBX,NBQ2)+\\
     \verb+COMMON/RGRIDF2/XX(NBX),Q2X(NBQ2)+\\
     \verb+COMMON/F2VAL/F2_DIS(NBX,NBQ2),+%
     \verb+F2_DIF(NBX,NBQ2),F2_PI(NBX,NBQ2)+\\ 
\verb+COMMON/RAPA/IPRO,IRUNA,IQ2,IRUNAEM,Q2START,W_Q2,OMEG2+ \\
\verb+COMMON/RAPGKI/YY,XEL,XPR,PT2H,SHH,T2GKI,XFGKI+ \\
\verb+COMMON/RESGAM/SCALQ2+\\
\verb+COMMON/SCALF/SCALFA+\\
\verb+COMMON/STRU/SCAL1,XPD1,SCAL2,XPD2+\\
\verb+COMMON/VMESON/IVM+\\

\section{Example Program}
\begin{verbatim}
      PROGRAM RGMAIN
	Implicit None
#include "rgfull.inc"
#include "rgdisdif.inc"
#include "rgluco.inc"
#include "rgludat1.inc"
#include "rgpara.inc"
#include "rgpypars.inc"
#include "rgdiffr.inc"
#include "rgscq2.inc"
#include "rglq2.inc"
#include "rghsunts.inc"
#include "rghsoptn.inc"
#include "rghscuts.inc"
#include "rghsvglp.inc"
	Integer Nevent
	Common/steer1/Nevent
      Integer N1,N2

	real timeleft
      Integer Minuts
      External Minuts
	External pydata

      Integer I,ISEED
C---initialise ARIADNE parameters, now done via block data
C---initialise PYTHIA 6 parameters, via pythia block data 
C     initialize random number generator
      ISEED = 213123
      ISEED = Iabs(MINUTS())
      N1 = 0
      N2 = 0
      CALL RM48IN(ISEED,N1,N2)
C---initialise RAPGAP parameters
      CALL GRAINI
C-- read in parameters from file 
      Call Steer
C-- change standard parameters of RAPGAP 	
	Call rapcha
C-- change standard parameters of HERACLES	
	Call hercha
C-- change standard parameters of \PYTHIA
	Call pytcha
C-- change standard parameters of ARIADNE
	Call aricha
	
C Initialize ARIADNE
      CALL ARINIT('RAPGAP')
C-- change standard parameters of ARIADNE
	Call aricha
C--- CALCULATE X SECTION
      CALL PTIME(' rapgap  ',1,0)
      CALL RAPGAP
C--- print x section
      CALL RAEND(1)
C--- event generation
      DO 10 I=1,Nevent
         CALL PTIME(' event  ',1,0)
c	   call timel(timeleft)
c	   if(timeleft.le.10.) then
c	      write(6,*) ' *** time limit reached. time left = ',timeleft
c            goto 123
c	   endif
         CALL EVENT
C--- user analysis routine
c         CALL ANALYS
C--- user analysis routine (hztool)
cc         CALL hzraana
      CALL PTIME(' event  ',2,0)
C---
   10 CONTINUE
123   Continue
C---PRINT NR OF GENERATED EVENTS
      CALL RAEND(20)	
C--- user terminate routine (hztool)
cc	CALL hzraend
	
      CALL PTIME(' rapgap  ',2,0)
      CALL PTIME('        ',2,99)
      STOP
      END
#include "../pythia-bkdata/pydata.f"
\end{verbatim}

In the distribution example programs are given also for the use with the 
\HZTOOL\ ~\cite{hztool2006}.
package. The source code is in directory {\tt examples} and the executable is
stored in the {\tt bin} directory. To access different subroutines form 
\HZTOOL\,
the file {\tt hzxxxx.inc} has to be changed accoringly.

The {\tt bin} directory contains also sample steering files, for different
processes.

To run \RAPGAP\ just type: \\
{\tt rapgap32 < steer.charm }\\
or using the \HZTOOL\ verison: \\
{\tt rapgaphztool < steer.charm }

\section{Installation Instructions}
\begin{verbatim}
Please note:
RAPGAP 3.2 contains the source code of HERACLES 4.63 (from .. )
RAPGAP 3.2 contains the source code of BASES/SPRING 5.1 (from ..)
RAPGAP 3.2 contains explicitly the block data from pythia.
           to change/update to newer version, please copy the block data
           pydata.f from the latest PYTHIA version into the directory
           rapgap-3.2/src/pythia-bkdata/pydata.f
1) Get the source

tar xvfz rapgap-3.2.tar.gz
cd rapgap-3.2

2) set environment variables for PYTHIA, CERNLIB and HZTOOL
example (Please change to the proper path of the libraries):
in csh: 
setenv PYTHIA "/Users/jung/jung/cvs/pythia6410"
setenv LHAPDF "/Users/jung/jung/cvs/lhapdf/lhapdf-5.7.0"

If you want to use ARIADNE, you need:

setenv ARIADNE "/Users/jung/jung/cvs/ariadne412"

If the HZTOOL interfae is wanted, you need in addition:

setenv CERN_LIBS "/sw/lib -lmathlib -lkernlib -lpacklib"
setenv HZTOOL "/Users/jung/jung/cvs/hztool-4/hztool-4.2"


in zsh:
export PYTHIA="/Users/jung/jung/cvs/pythia6410"
export LHAPDF="/Users/jung/jung/cvs/lhapdf/lhapdf-5.7.0"

If you want to use ARIADNE, you need:

export ARIADNE="/Users/jung/jung/cvs/ariadne412"

If the HZTOOL interfae is wanted, you need in addition:

export CERN_LIBS="/sw/lib -lmathlib -lkernlib -lpacklib"
export HZTOOL="/Users/jung/jung/cvs/hztool-4/hztool-4.2"

2) Generate the Makefiles
./configure --disable-shared --prefix=install-path

if you want to create an execuatble using HZTOOL 
./configure --disable-shared --prefix=install-path hzexe=yes

if you want to create an execuatble using ARIADNE
./configure --disable-shared --prefix=install-path ariadne=yes

To activate HepMC2, HepMC3 and/or Rivet interface use 

--with-hepmc2=/locationof/hepmc2/
--with-hepmc3=/locationof/hepmc3/
--with-rivet=/locationof/rivet/

3) Compile the binary and the documentation
make
  Install the program
make install

4) the execuatble is in "your-installation-directory"/bin
run it with:
install-path/bin/rapgap < install-path/share/steer_pp-bottom

5.) the documentation is in
install-path/share/rapgap32.pdf

6.) 
\end{verbatim}
\section{Acknowledgment}
I am grateful to T. Sj\"ostrand and G. Ingelman, the authors of \PYTHIA\ and 
\LEPTO\ for their permission to copy and use subroutines from their programs.
All the credit concerning initial state parton showers and remnant treatment
belongs to them.
\par
I want to thank all those who continue-sly used and checked the program.
Special thanks go to G.~Briskin, A.~Mehta and J.~Philipps. Without
their ideas, suggestions and comments and heavily use of the program, it
would not be like it is now.

%
%-----------------------------------------------------------------------
%

\section{Update History}
\begin{verbatim}
RAPGAP 3.2    
*::>         Version 3.2007  (Nov 2011)
*            prompt photon processes included
*      
*::>         Version 3.2006  (May 2011)
*            xf description in manual corrected (Radek Zlebcik)
*            eventpp: removed unnecessary boost to-from CM system (Radek Zlebcik)
*::>         Version 3.2003  (Sept 2010)
*            bug in heavy flavor selection in eleres corrected
*            bug in h1qcd/qcd_2006  on limit of iz,iq2 corrected
*            W/Z production included for pp
*            Higgs production included for pp
* 
*::>         Version 3.2002
*            bug in diffraction for pp causing energy momentum conservation problem corrected 
*            pp works also with new remnant treatment
*            Lambda_QCD now by default taken from PDF. Use ILAMQCD=0 to switch to steering 
*		
*::>         Version 3.2001
*            release for hepforge with auto Tools installation
*            For ep: HERACLES 4.6.6    
*            CERNLIB lite included
*            problem NFL_GAM in eleres > NFLAV...check and give proper return to program
*________________________________________________________________________
RAPGAP 3.1   
*::>         Version 3.1035
*            bug in lmeps for resloved diff scattering corrected
*		
*::>         Version 3.1034
*            lower cut for pdfs in h1qcd2006 removed, important for charm
*            f2c for diffraction:printout statements restricted
*		
*::>         Version 3.1033
*		 bug in rgmain: arinit overwrote changed PYTHIA/JESTET parameters
*
RAPGAP 3.1   
*::>         Version 3.1032
*            mstp(56) changed to mstp(55) for consitency with PYTHIA using LHAPDF
*            nhera mode changed to run with standard lhapdf
*            changes made for proton,pion,photon pdfs
*		
*::>         Version 3.1031
*            bug corrected in pion_xpq
*            now also pion pdf from LHAPDF or pdflib (h1qcd2006,h1qcdfit)
*            problem of very small scales and LHAPDF with HERACLES corrected (rystfu)
*		
*
*::>         Version 3.1029
*            new 2006 diff pdfs from H1 included
*            eleres now also for IHFLA=5 working
*		
*
*::>         Version 3.1027
*            LHApdf interface for both proton and photon pdfs included
*
*::>         Version 3.1026
*            bug of formula for ksi in eleqqf corrected: resulted in wrong
*            xsection for CC heavy quark production via IPRO=14
*
*::>         Version 3.1025
*            updated for heracles 
*
*::>         Version 3.1024
*            for charged current, bug in QCDMIX corrected, to have
*            proper flavor for heavy quarks for IFUL=1
*
*::>         Version 3.1023  
*            for charged current, bug in HERACLES interface corrected, to have
*            proper neutrino in final state
*
*::>         Version  3.1022
*            order of changes of PYHTIA and RAPGAPO params in  
*            h1 version changed .... to avoid overwritings ... grapga.F
*            bug for pp version corrected
*
*::>         Version 3.1021
*            Updated versions for UPINIT and UPEVNT 
*
*::>         Version 3.1019
* 		 PYTHIA block data are now called via external statement
*
*::>         Version 3.1020
*            steering files updated according to new frag. parameters
*            Bug in steer corrected: Nmax = 1000
*            Also RGUPINIT and RGUPEVNT added for LHA
*
*::>         Version 3.1014
*            bug with splitted variables in rapgap.f corrected
*            bug writing bspdf.dat instead of pdf.dat in rapgap.f corrected
*
*::>         Version 3.1013
*            pdiss ala diffvm included also for IPRO=30
*
*::>         Version 3.1012
*            pion exchange for neutrons added
*            ARIADNE now for all IPRO available (also for IPRO=15,18)
*
*::>         Version 3.1011
*            updated eleres for charm 
*            bug for xb < 0 in PYSSPA corrected

*::>         Version 3.1008
*            Proton Dissociation included, with routines from DIFFVM (B.List)
*            started by P. Thompson and Y. Coppens
*            Updated treatement of p-diss system acc. to Leszek Adamczyk
*            q-gluon-diq        
*
RAPGAP 3.0 (with PYTHIA6)
*::>         Collins NLO (order alphas) for F2 implemented (S. Schilling)
*
*::>          
RAPGAP 2.8 (with JETSET 7.4)
*
*::>          VERSION  2.08/19   01/08/2002 12.28.22
*             Bug in initialisation of RM48 found by Thomas Kluge. Thanks.
*
*::>          VERSION  2.08/18   16/01/2002 18.27.20
*             bug in rgsatrev corrected: PHIP missing
*
*::>          VERSION  2.08/17   24/10/2001 17.50.59
*             upper scale for final state ps for IPRO 18 updated. Was shat
*             before always !!!
*
*::>          VERSION  2.08/16   14/08/2001 08.27.06
*             bug in eleqqg corrected: 2pi in gamma flux, 2 in l integral
*             bug in partdh corrected: partons for qqg were mixed, caused frag
*             crash in case of charm
*             eleqq updated to use also unintegrated gluon
*
*::>          VERSION  2.08/15   25/06/2001 18.18.51
*             ppbar for incl and diffraction included
*
*::>          VERSION  2.08/14   14/05/2001 13.53.48
*             some double defined subroutines removed and ariadne common blocks
*             and PYPARS updated.
*
*::>          VERSION  2.08/11   08/03/2001 11.00.31
*             small bug in PARTDF corrected, for p* momentum with m ne mp
*
*::>          VERSION  2.08/10   07/02/2001 12.02.40
*             DUDBRB changed to LUDBRB in AREXEC. Bug found by Taro Yamashita
*
*::>          VERSION  2.08/09   17/12/2000 17.00.34
*             In partdi,partdf for heracles mode cut on q2min < q2max removed
*             because events were rejected. Also in parti for HERACLES WEIGHT=1
*             added, all by request from A. KAppes and K.Long.
*             New version of h1qcdfit parameterisation: Redone charm such that
*             charm pdf = BGF *gluon
*             New double precision draprn (using RM48 of CERNLIB). Problems
*             found by Taro in y distribution for IPRO=20. Now is ok.
*
*::>          VERSION  2.08/07   31/07/2000 16.58.11
*             eleqqg (2 gluon exchange) updated for heavy quarks (A. Kyrieleis)
*
*             eleres charge for top corrected
*
*::>          VERSION  2.08/05   30/03/2000 17.28.20
*
*             rapgap and event changed because of a bug for ipro1200 and pi exchange
*             SATRAP now includes also charm
*
*
*::>          VERSION  2.08/03   10/12/99 12.21.44
*             rgmain bug corrected
*
*::>          VERSION  2.08/02   04/11/99 16.39.29
*             SATRAP (IPRO=30) now included (works also with HERACLES interface
*             IPRO=3000).
*             "Bug" in phi angle of scattered proton for diffraction corrected
*
*::>          VERSION  2.08/01   23/06/99 14.31.37
*             new treatment for diffraction and me mixing...
*
*::>          VERSION  2.08/00   15/06/99 17.31.15
*
*             Bug for IPRO=1200,1400 in diffractive mode with IFULL=1 corrected
*             eta distribution of partons from matrix elements were wrong
*
*             Bug in xpom distribution for diffraction with IFULL=1 corrected
*
*             improved calculation of ME contribution in diffraction, taken into account
*             kinematic limits from pt2cut
*
*             Nr of flavours in structure function now also treated correctly for the
*             HERACLES interface
*
*             test release for IPRO=30, diffraction ala saturation model of M. Wuesthoff
*             coded by H. Kowalski
*
*________________________________________________________________________
*
*::>          VERSION  2.07/01   23/03/99 09.07.40
*             heavy flavor excitation in res. photon processes added
*             better treatment of ME mixing in diffr. processes
*
*::>          VERSION  2.07/00   22/01/99 09.30.01
*             pi flux of Holtmann now default
*
*::>          VERSION  2.06/51   10/10/98 18.30.56
*             LEPTOU changed to RAPTOU to avoid confusion with DJANGOH
*
*::>          VERSION  2.06/50   30/09/98 08.19.42
*             some default settings updated
*
*::>          VERSION  2.06/49   29/09/98 18.46.00
*             bug for heavy quarks in resolved photon processes corrected
*
*::>          VERSION  2.06/47   19/09/98 19.41.19
*             heavy quark ME's for resolved photons inplemented
*             bug for GADAP2 integration of ME corrected
*
*::>          VERSION  2.06/44   01/08/98 18.19.48
*             heracles 4.5 and heracles 4.6 included via SEL=HERACLES46
*
*::>          VERSION  2.06/43   24/07/98 15.25.29
*             bugs in dimensions for x() corrected
*
*::>          VERSION  2.06/42   23/07/98 15.39.31
*             dimension for x() increased to 20 (necessary for qqg)
*
*::>          VERSION  2.06/41   22/07/98 10.23.45
*             QCDGRID speede up by usage of grid for wmax
*
*::>          VERSION  2.06/38   15/07/98 22.50.21
*             heracles fl included
*
*::>          VERSION  2.06/37   30/06/98 22.30.23
*             Bug in intrinsic k_t for resolved photons corrected (PYREMN)
*
*::>          VERSION  2.06/36   26/06/98 08.36.26
*             resolved photon BFKL qq -> qq added acc. Cox/Forshaw hep-ph9800506
*
*::>          VERSION  2.06/34   17/06/98 23.08.14
*
*             writing grid file for pdf's included --> faster when same param.
*             processes IPRO=21 included qq 2gluon exchange
*             processes IPRO=20 included qqg 2gluon exchange
*
*::>          VERSION  2.06/33   14/05/98 08.47.12
*             bug when running 1200 with IGRID=1 corrected
*
*::>          VERSION  2.06/32   05/05/98 14.50.19
*             clebsch/cordon coeff for pion exchange corrected in RAT2DI
*
*::>          VERSION  2.06/31   26/03/98 10.34.34
*             2nd attempt for qqg a la Bartels/Wuesthoff
*
*::>          VERSION  2.06/30   18/03/98 13.30.56
*             intrinsic pt for photon and proton changeable by user
*             bug corrected for photon pdfs using pdflib
*
*::>          VERSION  2.06/29   16/03/98 10.08.13
*             1st attempt to include res. gamma also for diffraction
*
*::>          VERSION  2.06/28   13/03/98 12.21.39
*             NFQCDC added: select 3 or more flavours for QCDC
*
*::>          VERSION  2.06/27   09/02/98 11.56.04
*             1st attempt for qqg a la Bartels/Wuesthoff
*
*::>          VERSION  2.06/26   24/01/98 17.15.29
*             fixed problem in phi asym. for hard diffraction
*
*::>          VERSION  2.06/25   14/01/98 20.59.01
*             KPF=KPA added for vm prod in partdf
*
*::>          VERSION  2.06/24   14/01/98 19.40.53
*             wuesthoff new parametrisation included (ANL-HEP-PR 97-03)
*             selected via NG=NPOM=40
*
*::>          VERSION  2.06/23   29/12/97 19.36.49
*             podiff improved.....
*
*::>          VERSION  2.06/22   15/12/97 11.52.17
*             weighting for diffr. with ng>0 improved
*
*::>          VERSION  2.06/21   09/12/97 16.58.25
*             bug in arexec for diffractive events corrected
*
*::>          VERSION  2.06/20   26/11/97 16.23.35
*             QCDMIX corrected when using grid
*
*::>          VERSION  2.06/19   24/11/97 15.40.20
*             bug in lmeps for QED radiation fixed
*
*::>          VERSION  2.06/17   04/11/97 16.26.19
*             Subroutine EVENT changed so that 1200(nrad) and 12 give same results
*             for diffraction
*
*::>          VERSION  2.06/16   31/10/97 15.11.16
*             lots of changes in PYSSPA for res. photons needed
*             rearranged switches to be more transparent....
*             included option for angluar ordering in PYSSPA
*
*::>          VERSION  2.06/13   20/10/97 12.22.59
*             bug in small mass collapsing for charm in LMEPS corrected
*
*::>          VERSION  2.06/12   18/10/97 15.15.55
*             2nd attempt to IPS in resolved photon
*
*::>          VERSION  2.06/11   18/10/97 00.46.47
*             first attempt to IPS in resolved photon
*
*::>          VERSION  2.06/10   16/10/97 12.08.35
*             NFLAV added to f2md
*
*::>          VERSION  2.06/09   11/10/97 14.09.07
*             IPRO=15 PHIGKI changed to region 0 - 2pi
*
*::>          VERSION  2.06/08   10/10/97 17.17.42
*             bug NIA1 used in AREXEC corrected
*
*::>          VERSION  2.06/07   10/10/97 12.23.08
*             check on IDISDIF in rapgap improved....
*
*::>          VERSION  2.06/06   10/10/97 10.37.30
*             bug corrected: mixing F2 and F2D for NG<0
*
*::>          VERSION  2.06/05   25/09/97 16.31.35
*             initial particle code for res. gamma corrected
*
*::>          VERSION  2.06/04   23/09/97 10.14.00
*             initial parton for resolved gamma in LMEPS corrected
*
*::>          VERSION  2.06/03   14/09/97 16.57.59
*             mismatch in dimensions in XXI and Q2I in rapgap cured
*
*::>          VERSION  2.06/01   21/07/97 16.11.46
*             angular cut for HERACLES included in hercl
*             to be used together with HERCALES version 4401
*
*::>          VERSION  2.06/00   18/07/97 13.35.24
*             RAPGAP for DIS and DIF including resolved photons for DIS
*
*::>          VERSION  2.05/28   16/07/97 10.42.54
*             HERACL, XMAX=1-XF for diffraction introduced to avoid
*             wispy y%q2 distribution....
*
*::>          VERSION  2.05/26   07/07/97 16.11.31
*             bugs in LMEPS for small systems corrected
*
*::>          VERSION  2.05/23   30/06/97 12.08.49
*             sasgam was not correctly included in car file
*
*::>          VERSION  2.05/22   29/06/97 13.02.39
*
*             PYSSPA: explicit lower limit for soft gluon rad. included
*                     --> problem with energy momentum cons. solved
*             Q2SUPP: parameter for HERACLES low Q2 suppression included
*             SCALQ2: parameter for scale/Q2 for resolved photons in DIS incl.
*             IHFLA : flavour code for heavy flavor (IPRO=14,1400) included
*             SCALFA: factor for scale Q2Q included, to switch from
*                     Q2Q --> SCALFA*Q2Q
*             LMEPS:  changes from G. Briskin for low mass diff. included
*             LUKFDI: changes from G.Briskin for low mass states included
*
*
*::>          VERSION  2.05/20   16/06/97 16.13.21
*             NPOM = -3 only pomeron
*             NPOM = -4 only meson
*
*::>          VERSION  2.05/19   09/06/97 09.02.25
*             scale pT**2 instead of 4*pt**2,
*             lambda consistently set also for PS
*             for HERACLES always fixed alpha_em
*
*::>          VERSION  2.05/16   08/06/97 19.34.46
*             alphas for diffr. was 0 in version 2.05 , corrected
*
*::>          VERSION  2.05/15   05/06/97 15.59.54
*             now QCDCOMPTON only for light quarks
*             some pause statements removed....
*
*::>          VERSION  2.05/13   26/05/97 16.47.48
*             ftncheked --> common mismatches removed
*
*::>          VERSION  2.05/12   16/05/97 16.28.40
*             pt2 scale now 4*pt2 instead of pt2
*
*::>          VERSION  2.05/11   15/05/97 13.30.48
*             w02 in partdi corrected
*
*::>          VERSION  2.05/09   12/05/97 09.38.57
*             res gamma in DIS hopefully correctly implemented
*
*::>          VERSION  2.05/07   30/04/97 17.46.55
*             parameters for alpha_s changed, MSTU(112) and PARU(112)
*             set properly to                 N_flavours    lambda(N_flavours)
*
*::>          VERSION  2.05/06   23/04/97 07.13.59
*             add virtual photon structure parametrisations
*
*::>          VERSION  2.05/05   29/03/97 13.47.29
*             matrix elements for resolved checked
*
*::>          VERSION  2.05/04   10/03/97 10.47.35
*             improved color flow for resolved gamma's
*
*::>          VERSION  2.05/02   06/03/97 20.32.47
*             small bugs corrected, BGF part for heracles included via grid
*
*::>          VERSION  2.05/01   06/03/97 11.18.21
*             bugs of color connection corrected
*
*::>          VERSION  2.05/00   05/03/97 08.10.38
*             resolved photon for DIS included via IPRO = 18
*
*::>          VERSION  2.04/05   01/03/97 14.59.20
*             BGF for bb_bar added and matrix element calculation
*             for alpha_s with different scale than Q2 added
*             --> change in pqcdi and pqcd added for scale pt**2
*
*::>          VERSION  2.04/04   24/01/97 10.47.35
*             test version for charged current (INTER=2)
*
*::>          VERSION  2.04/03   20/01/97 08.45.34
*             test version
*
*::>          VERSION  2.04/02   25/12/96 23.10.05
*             bug corrected in partdi partdf for INTER= 2

*________________________________________________________________________
*
*::>          VERSION  2.04/00   23/12/96 13.56.36
*             charged current interactions (W exchange) included
*________________________________________________________________________
*
*::>          VERSION  2.03/13   18/10/96 09.03.33
*             IVM in partdf corrected
*
*::>          VERSION  2.03/12   14/10/96 14.44.28
*             change in PYSSPA  to account for new sea treatment a la LEPTO 65
*
*::>          VERSION  2.03/11   11/10/96 12.46.04
*             change in PYSSPA to ensure energy and momentum conservation
*             for real particle
*             change in RAPGAP to ensure original values of IPY(13) and IPY(14)
*
*::>          VERSION  2.03/10   04/10/96 17.25.26
*             Weighing modified to weight as a function of electron
*             method Q2. 
*
*::>          VERSION  2.03/08   03/10/96 20.24.10
*             GRAWQ2 added for weighting in Q2.
*             Directory H1QCD added for H1 QCD/Phenomenological
*             parameterisations.
*
*::>          VERSION  2.03/07   27/09/96 09.42.02
*             PYSSPA changed for xpq=0
**
*::>          VERSION  2.03/05   06/09/96 13.50.46
*             debug statement in event implemented
*
*::>          VERSION  2.03/04   05/09/96 18.58.09
*             Bugs for HERACLES with user defined parton density etc corrected
*             HARD POM NG=40,41,42 should work now
*::>          VERSION  2.03/03   22/08/96 20.17.55
*             cosmetics
*
*::>          VERSION  2.03/02   18/08/96 17.46.10
*             small bugs in hardpom corrected
*
*::>          VERSION  2.03/01   16/08/96 09.40.35
*             Kt for ALM only for IPRO=12 and add error summary added
*
*::>          VERSION  2.03/00   14/08/96 14.36.32
*             kt for ALM added via switch IALMKT=1
*
*::>          VERSION  2.02/04   14/08/96 14.32.24
*             kt for ALM added via switch IALMKT=1 otherwise =0
*
*::>          VERSION  2.02/02   29/07/96 11.17.05
*             NG = 21 NPOM =21 for pi0 exchange added
*             IPRO = 1400 for cc_bar with HERACLES added
*
*::>          VERSION  2.00/30   01/01/96 17.16.30
*             NIK ZAK included
*             PARTDF corrected for correct treatment of beta,x_pom
*             switch for p diss treatment included
*             info for ME added in common/MEINFO/
*
*::>          VERSION  2.00/29   03/12/95 12.02.34
*             Heracles - pion exchange bug corrected in RAPGAP: PIMAX missing
*
*::>          VERSION  2.00/27   22/11/95 16.35.16
*             treatment of vm production in connection with ariadne corrected
*
*::>          VERSION  2.00/26   21/11/95 12.30.17
*             HERACLES opt in PARTDF updated for initial state PS
*
*::>          VERSION  2.00/24   19/11/95 21.43.54
*             IFULL=1,IQ2=4 and IPRO=12 now default.
*             ARIADNE opt with same ev.record as MEPS option.
*
*::>          VERSION  2.00/22   09/11/95 13.39.40
*             in lmeps corrected to run with final state PS
*             mother of rad gamma included
*             GTR bank filled for beam part. and rad gamma acc. H1 convention
*
*::>          VERSION  2.00/19   07/11/95 14.28.29
*             bug for ariadne corrected.....
*             bug for x section if IFUL=1 corrected
*             new version for p-diss
*
*::>          VERSION  2.00/18   11/10/95 09.54.37
*             multiple definition of common PYINT1; corrected
*
*::>          VERSION  2.00/17   10/10/95 14.23.58
*             bugs fixed for HERACLES diffraction and ME option
*
*::>          VERSION  2.00/15   06/10/95 15.48.06
*             bugs fixed for HERACLES diffraction and ME option
*
*::>          VERSION  2.00/14   04/10/95 09.57.56
*             typing error for QCDC corrected  (thanks to Nils and Gunilla)
*
*::>          VERSION  2.00/12   20/09/95 08.45.47
*             phi asymmetry corrected, wrong frame
*
*::>          VERSION  2.00/10   10/09/95 14.22.11
*	        selection for light and heavy vectormesons included: IVM =0/1/443
*
*::>          VERSION  2.00/09   09/09/95 16.56.13
*             prodiff changed: intrinsic pt added, pz_pom subtr.
*
*::>          VERSION  2.00/08   31/08/95 11.23.17
*             bug for phi asymmetries in eleqqf and eleqcdc corrected
*             bug corrected in partdf causing energy momentum nonconserv.
*
*::>          VERSION  2.00/07   28/07/95 15.36.31
*             endless loop fixed in PYREMN
*
*::>          VERSION  2.00/06   26/07/95 15.18.07
*             small bug for heracles with THEMIN corrected
*
*::>          VERSION  2.00/05   23/07/95 15.20.52
*             no cut necessary: AM(1) was not defined in PARTDI and PARTDF
*
*::>          VERSION  2.00/04   20/07/95 10.10.11
*             tried with a cut in eleqcdc introduced to avoid divergency zq-->1
*
*::>          VERSION  2.00/03   04/07/95 16.01.00
*             full formula for QCD compton added including phi asymmetries
*
*::>          VERSION  2.00/02   16/06/95 08.51.03
*             some printout reduced in partdf and pyremn
*
*::>          VERSION  2.00/01   11/05/95 11.20.53
*             HERACLES interfaced for DIS and DIF
*
*
*::>          VERSION  2.00/00   19/04/95 20.21.57
*             new release
*
*::>          VERSION  1.04/03   23/03/95 14.44.49
*             Heracles interface added
*
*::>          VERSION  1.04/00   19/01/95 19.53.53
*
*             initial and final state parton shower included
*             a la LEPTO.
*             mixing of lowest order and hgher order QCD possible
*             also DIS and DIFF possible via IDIR=1 ==> DIS
*                                            IDIR=0 ==> DIF

*
*::>          VERSION  1.04/03   23/03/95 14.44.49
*             Heracles interface added
*
*::>          VERSION  1.04/00   19/01/95 19.53.53
*
*             initial and final state parton shower included
*             a la LEPTO.
*             mixing of lowest order and hgher order QCD possible
*             also DIS and DIFF possible via IDIR=1 ==> DIS
*                                            IDIR=0 ==> DIF
*
*::>          VERSION  1.03/04   31/10/94 10.16.50
*             include process IPRO=100
*             gamma + pomeron --> rho + pomeron
*
*             YMAX and Q2MAX cuts implemented
*
*::>          VERSION  1.03/02   21/04/94 11.36.23
*             now also cut on Minimum scattering angle of electron
*             THEMI included
*
*
*::>          VERSION  1.03/01   03/04/94 16.46.19
*             CMZ version of RAPGAP
*


\end{verbatim}

\begin{theindex}

  \item ALPH, 28
  \item ALPHP, 27
  \item ALPHS, 28
  \item AM, 28
  \item AVGI, 27

  \indexspace

  \item C1, 26
  \item Cg, 26

  \indexspace

  \item EPSP, 27

  \indexspace

  \item IALMKT, 27
  \item IALP, 24
  \item IBS, 23
  \item IDIR, 23
  \item IDISDIF, 23
  \item IFPS, 23
  \item IFULL, 23
  \item IGLU, 27
%  \item ILHA, 24
  \item INTER, 23
  \item IORD, 24
  \item IPRO, 22
  \item IQ2, 25
  \item IQCDGRID, 23
  \item Iqqg, 26
  \item IREM, 27
  \item IRPA, 24
  \item IRPB, 24
  \item IRPC, 24
  \item IRPD, 24
  \item IRPE, 24
  \item IRPF, 24
  \item IRPG, 24
  \item IRUNA, 24
  \item IRUNAEM, 24
  \item ISEMIH, 23
  \item ISOG, 24
  \item ITIM, 24
  \item IVM, 27

  \indexspace

  \item KBEAM, 27
  \item KT1, 24
  \item KT2, 24

  \indexspace

  \item MSTP(51), 25
  \item MSTP(52), 25
  \item MSTP(56), 25

  \indexspace

  \item NF1,NF2, 28
  \item NFLAV, 22
  \item NFLQCDC, 22
  \item NFRAG, 23
  \item NFT, 28
  \item NG, 25
  \item NIA1,NIA2, 28
  \item NIN, 28
  \item NOUT, 28
  \item NPOM, 26

  \indexspace

  \item OMEG2, 24

  \indexspace

  \item PBEAM, 27
  \item PEPS, 27
  \item PHIGKI, 28
  \item PI, 28
  \item PIN, 22
  \item PLEPIN, 22
  \item PRN2, 27
  \item PT2CUT, 23
  \item PT2H, 27

  \indexspace

  \item Q2, 27
  \item Q2MAX,Q2MIN, 27
  \item Q2Q, 28
  \item Q2SUPP, 23
  \item QMA, 22
  \item QMI, 22

  \indexspace

  \item RN2, 27

  \indexspace

  \item SCAL1,SCAL2, 28
  \item SCALFA, 25
  \item SCALQ2, 24
  \item SD, 27
  \item SHAT, 28
  \item SHH, 28
  \item SIN2W, 23
  \item SSS, 27

  \indexspace

  \item T2GKI, 28
  \item T2MAX, 27
  \item THEMA,THEMI, 22

  \indexspace

  \item XDP1,XPD2, 28
  \item XEL, 27
  \item XF, 27
  \item XFGKI, 28
  \item XMAX,XMIN, 27
  \item XMW2, 23
  \item XPGKI, 28
  \item XPR, 27

  \indexspace

  \item YMA, 22
  \item YMAX,YMIN, 27
  \item YMI, 22
  \item YY, 27

  \indexspace

  \item ZQGKI, 28

\end{theindex}

\providecommand{\etal}{et al.\xspace}
\providecommand{\href}[2]{#2}
\providecommand{\coll}{Coll.}
\catcode`\@=11
\def\@bibitem#1{%
\ifmc@bstsupport
  \mc@iftail{#1}%
    {;\newline\ignorespaces}%
    {\ifmc@first\else.\fi\orig@bibitem{#1}}
  \mc@firstfalse
\else
  \mc@iftail{#1}%
    {\ignorespaces}%
    {\orig@bibitem{#1}}%
\fi}%
\catcode`\@=12
\begin{mcbibliography}{10}

\bibitem{pythia64}
T.~Sjostrand, S.~Mrenna, and P.~Skands,
\newblock JHEP{} {\bf 05},~026~(2006).
\newblock \href{http://www.arXiv.org/abs/hep-ph/0603175}{{\tt
  hep-ph/0603175}}\relax
\relax
\bibitem{Herwig}
G.~Marchesini, B.~Webber, G.~Abbiendi, I.~Knowles, M.~Seymour, and L.~Stanco,
\newblock Comp. Phys. Comm.{} {\bf 76},~465~(1992).
\newblock Hep-ph/9912396\relax
\relax
\bibitem{bases}
S.~Kawabata,
\newblock Comp. Phys. Comm.{} {\bf 88},~309~(1995)\relax
\relax
\bibitem{Whalley:2005nh}
M.~R. Whalley, D.~Bourilkov, and R.~C. Group~(2005).
\newblock \href{http://www.arXiv.org/abs/hep-ph/0508110}{{\tt
  hep-ph/0508110}}\relax
\relax
\bibitem{Ingelman_Prytz}
G.~Ingelman and K.~Prytz,
\newblock Z. Phys.{} {\bf C58},~285~(1993)\relax
\relax
\bibitem{diff_ws_sum}
H.~Abramowicz, J.~Bartels, L.~Frankfurt, and H.~Jung,
\newblock {\em {D}iffractive {H}ard {S}cattering. {S}ummary report of the
  working group},
\newblock in {\em Proc. of the Workshop on Future Physics at HERA},
  eds.~A.~\mbox{De Roeck}, G.~Ingelman, and R.~Klanner, p. 635.
\newblock \mbox{DESY}, Hamburg, 1996\relax
\relax
\bibitem{IS}
G.~Ingelman and P.~Schlein,
\newblock Phys. Lett. {\bf B}{} {\bf 152},~256~(1985)\relax
\relax
\bibitem{Strenga}
K.~Streng,
\newblock {\em Hard qcd scatterings in diffractive reactions at hera},
\newblock in {\em Proc.\ of the Workshop Physics at HERA}, ed.~R.~Peccei.
\newblock Hamburg, 1987\relax
\relax
\bibitem{Strengb}
K.~Streng,
\newblock {\em Hard qcd scatterings in diffractive reactions at hera}, 1988.
\newblock \mbox{CERN-TH} 4949 (1988)\relax
\relax
\bibitem{Berger_Collins}
E.~Berger, J.~Collins, D.~Soper, and G.~Sterman,
\newblock Nucl. Phys. {\bf B}{} {\bf 286},~704~(1987)\relax
\relax
\bibitem{Ingelman_Brunia}
P.~Bruni and G.~Ingelman,
\newblock Phys. Lett. {\bf B}{} {\bf 311},~317~(1993)\relax
\relax
\bibitem{Ingelman_Brunib}
P.~Bruni and G.~Ingelman,
\newblock {\em Diffractive hard scattering at e p and p anti-p colliders},
\newblock in {\em Proc. of the EPS International High Energy Physics
  Conference}, eds.~J.~Carr and M.~Perrottet.
\newblock Editions Frontieres, Marseille, France, 22-28 Jul, 1993.
\newblock \mbox{DESY} 93-187\relax
\relax
\bibitem{Ingelman_Brunic}
P.~Bruni, G.~Ingelman, and A.Solano,
\newblock {\em Diffractively produced hadronic final states and the pomeron
  structure},
\newblock in {\em Proc.\ of the Workshop on Physics at HERA, Vol. 1},
  eds.~W.~Buchm\"uller and G.~Ingelman, Vol. 311, p. 363.
\newblock \mbox{DESY}, Hamburg, 1991\relax
\relax
\bibitem{DL_84}
A.~Donnachie and P.~Landshoff,
\newblock Nucl. Phys. {\bf B}{} {\bf 244},~322~(1984)\relax
\relax
\bibitem{H1_F2D3_97}
\mbox{H1} Collaboration; C. Adloff~et al.,
\newblock Z. Phys. {\bf C}{} {\bf 76},~613~(1997)\relax
\relax
\bibitem{H1_F2D3_06}
H.~Collaboration,
\newblock {\em Measurement and qcd analysis of the diffractive deep-inelastic
  scattering cross section at hera}, 2006.
\newblock Hep-ex/0606004\relax
\relax
\bibitem{Wusthoff}
M.~Wusthoff,
\newblock {\em Photon diffractive dissociation in deep inelastic scattering},
  1995.
\newblock \mbox{PhD thesis}, \mbox{DESY-95-166}\relax
\relax
\bibitem{Diehl1}
M.~Diehl,
\newblock {\em Diffraction in electron - proton collisions}, 1996.
\newblock \mbox{PhD thesis}\relax
\relax
\bibitem{Diehl2}
M.~Diehl,
\newblock Z. Phys. {\bf C}{} {\bf 76},~499~(1997).
\newblock Hep-ph/9610430\relax
\relax
\bibitem{Bartels_dijet_ws}
J.~Bartels, C.~Ewerz, H.~Lotter, M.~Wusthoff, and M.~Diehl,
\newblock {\em {Q}uark - {A}ntiquark {J}ets in {DIS} {D}iffractive
  {D}issociation},
\newblock in {\em Proc. of the Workshop on Future Physics at HERA},
  eds.~A.~\mbox{De Roeck}, G.~Ingelman, and R.~Klanner.
\newblock \mbox{DESY}, Hamburg, 1996.
\newblock Hep-ph/9609239\relax
\relax
\bibitem{Bartels_jets}
J.~Bartels, H.~Lotter, and M.~W\"usthoff,
\newblock Phys. Lett. {\bf B}{} {\bf 379},~239~(1996).
\newblock \mbox{ERRATUM}-ibid. {\bf B 382} (1996) 449, hep-ph/9602363\relax
\relax
\bibitem{Bartels_asym}
J.~Bartels, C.~Ewerz, H.~Lotter, and M.~Wusthoff,
\newblock Phys. Lett. {\bf B}{} {\bf 386},~389~(1996).
\newblock Hep-ph/9605356\relax
\relax
\bibitem{GRVa}
M.~Gl\"uck, E.~Reya, and A.~Vogt,
\newblock Z. Phys. {\bf C}{} {\bf 53},~127~(1992)\relax
\relax
\bibitem{GRVb}
M.~Gl\"uck, E.~Reya, and A.~Vogt,
\newblock Phys. Lett. {\bf B}{} {\bf 306},~391~(1993)\relax
\relax
\bibitem{Bartels_qqga}
J.~Bartels, H.~Jung, and M.~Wusthoff,
\newblock Eur. Phys. J. {\bf C}{} {\bf 11},~111~(1999).
\newblock Hep-ph/9903265\relax
\relax
\bibitem{Bartels_qqg-charm}
J.~Bartels, H.~Jung, and A.~Kyrieleis,
\newblock Eur. Phys. J. {\bf C}{} {\bf 24},~555~(2002).
\newblock Hep-ph/0204269\relax
\relax
\bibitem{jung-dis04}
H.~Jung,
\newblock {\em Un-integrated parton density functions in ccfm}, April 2004.
\newblock \mbox{DIS 2004}, Strbsk\'e Pleso,~Slovakia, hep-ph/0411287\relax
\relax
\bibitem{GRV95}
M.~Gl\"uck, E.~Reya, and A.~Vogt,
\newblock Z. Phys. {\bf C}{} {\bf 67},~433~(1995)\relax
\relax
\bibitem{Bluemlein}
J.~Bl\"umlein,
\newblock {\em On the $k_t$ dependent gluon density of the proton},
\newblock in {\em Proc. of the Workshop on Deep Inelastic Scattering and QCD},
  eds.~J.~Laporte and Y.~Sirois.
\newblock 1995.
\newblock \mbox{DESY} 95-121 and hep-ph/9506403\relax
\relax
\bibitem{baranov_zotov_1999}
S.~Baranov and N.~Zotov,
\newblock Phys. Lett. {\bf B}{} {\bf 458},~389~(1999)\relax
\relax
\bibitem{baranov_zotov_2000}
S.~Baranov and N.~Zotov,
\newblock Phys. Lett. {\bf B}{} {\bf 491},~111~(2000)\relax
\relax
\bibitem{martin_stasto}
J.~Kwiecinski, A.~Martin, and A.~Stasto,
\newblock Phys. Rev. {\bf D}{} {\bf 56},~3991~(1997)\relax
\relax
\bibitem{wuesthoff_golec-biernat}
K.~Golec-Biernat and M.~Wusthoff,
\newblock Phys. Rev. {\bf D}{} {\bf 60},~114023~(1999).
\newblock Hep-ph/9903358\relax
\relax
\bibitem{martin_kimber}
M.~A. Kimber, A.~D. Martin, and M.~G. Ryskin,
\newblock Phys. Rev.{} {\bf D63},~114027~(2001).
\newblock \href{http://www.arXiv.org/abs/hep-ph/0101348}{{\tt
  hep-ph/0101348}}\relax
\relax
\bibitem{Buchmuller_Hebecker_Mcdermott}
W.~Buchmuller, M.~McDermott, and A.~Hebecker,
\newblock Nucl. Phys. {\bf B}{} {\bf 487},~283~(1997).
\newblock Hep-ph/9607290\relax
\relax
\bibitem{NikZak}
N.~Nikolaev and B.~Zakharov,
\newblock Z. Phys. {\bf C}{} {\bf 53},~331~(1992)\relax
\relax
\bibitem{satrap}
H.~Kowalski and M.~Wusthoff,
\newblock {\em Satrap model for diffractive dis processes}.
\newblock Prepared for 8th International Workshop on Deep Inelastic Scattering
  and QCD (DIS 2000), Liverpool, England, 25-30 Apr 2000\relax
\relax
\bibitem{HERACLESa}
A.~Kwiatkowski, H.~Spiesberger, and H.-J. M\"ohring,
\newblock {\em Heracles 4.1 - an event generator for $ep$ interactions at hera
  including radiative processes},
\newblock in {\em Proc.\ of the Workshop on Physics at HERA, Vol. 3},
  eds.~W.~Buchm\"uller and G.~Ingelman, p. 1294.
\newblock 1991\relax
\relax
\bibitem{HERACLESb}
H.~Spiesberger,
\newblock {\em \mbox{HERACLES and DJANGOH} - Event generation of $ep$
  interactions at HERA including radiative processes. version 4.6.6}, Aug 2005.
\newblock
  \verb+http://wwwthep.physik.uni-mainz.de/~hspiesb/djangoh/djangoh.html+\relax
\relax
\bibitem{sophia}
A.~Mucke, R.~Engel, J.~P. Rachen, R.~J. Protheroe, and T.~Stanev,
\newblock Comput. Phys. Commun.{} {\bf 124},~290~(2000).
\newblock \href{http://www.arXiv.org/abs/astro-ph/9903478}{{\tt
  astro-ph/9903478}}\relax
\relax
\bibitem{DGLAPa}
V.~Gribov and L.~Lipatov,
\newblock Sov. J. Nucl. Phys.{} {\bf 15},~438 and 675~(1972)\relax
\relax
\bibitem{DGLAPb}
L.~Lipatov,
\newblock Sov. J. Nucl. Phys.{} {\bf 20},~94~(1975)\relax
\relax
\bibitem{DGLAPc}
G.~Altarelli and G.~Parisi,
\newblock Nucl. Phys. {\bf B}{} {\bf 126},~298~(1977)\relax
\relax
\bibitem{DGLAPd}
Y.~Dokshitser,
\newblock Sov. Phys. JETP{} {\bf 46},~641~(1977)\relax
\relax
\bibitem{Collins-collfac}
J.~C. Collins and X.-M. Zu,
\newblock JHEP{} {\bf 06},~018~(2002).
\newblock Hep-ph/0204127\relax
\relax
\bibitem{Collins-collfac2}
J.~C. Collins and X.~Zu~(2004).
\newblock Hep-ph/0411332\relax
\relax
\bibitem{sschilling}
S.~Schilling,
\newblock {\em Implementation of bgf-processes in monte carlo generators for
  electron proton scattering}.
\newblock DESY-THESIS-2000-040\relax
\relax
\bibitem{Chyla_res_gamma}
J.~Chyla and J.~Cvach,
\newblock {\em {V}irtual photon structure from jet production at {HERA}},
\newblock in {\em Proc. of the Workshop on Future Physics at HERA},
  eds.~A.~\mbox{De Roeck}, G.~Ingelman, and R.~Klanner.
\newblock Hamburg, 1996\relax
\relax
\bibitem{RAPGAP32}
H.~Jung,
\newblock {\em \mbox{T}he RAPGAP Monte Carlo for Deep Inelastic Scattering,
  version 3.2}.
\newblock DESY Hamburg, 2007.
\newblock \verb+http://www.desy.de/~jung/rapgap/+\relax
\relax
\bibitem{GRS}
M.~Gl\"uck, E.~Reya, and M.~Stratman,
\newblock Phys. Rev. {\bf D}{} {\bf 54},~5515~(1996)\relax
\relax
\bibitem{Sasgam}
G.~Schuler and T.~Sj\"ostrand,
\newblock Phys. Lett. {\bf B}{} {\bf 376},~193~(1996)\relax
\relax
\bibitem{Drees_Godbole}
M.~Drees and R.~Godbole,
\newblock Phys. Rev. {\bf D}{} {\bf 50},~3124~(1994)\relax
\relax
\bibitem{Mirkes_nlo}
E.~Mirkes and D.~Zeppenfeld,
\newblock Phys. Lett. {\bf B}{} {\bf 380},~105~(1996)\relax
\relax
\bibitem{catani_nlo}
S.~Catani and M.~Seymour,
\newblock {\em {NLO QCD} calculations in {DIS} at {HERA} based on the dipole
  formalism},
\newblock in {\em Proc. of the Workshop on Future Physics at HERA},
  eds.~A.~\mbox{De Roeck}, G.~Ingelman, and R.~Klanner.
\newblock Hamburg, 1996\relax
\relax
\bibitem{Kramer_res_gamma}
M.~Klasen, G.~Kramer, and B.~Potter,
\newblock Eur. Phys. J. {\bf C}{} {\bf 1},~261~(1998).
\newblock Hep-ph/9703302\relax
\relax
\bibitem{MEPS}
G.~Ingelman,
\newblock {\em Lepto 6.1 the lund monte carlo for deep inelastic lepton-nucleon
  scattering},
\newblock in {\em Proc. of the Workshop Physics at HERA (1991) Vol. 3, 1366},
  eds.~W.~Buchm\"uller and G.~Ingelman.
\newblock \mbox{DESY}, Hamburg, 1991\relax
\relax
\bibitem{Jetseta}
T.~Sj\"ostrand,
\newblock Comp. Phys. Comm.{} {\bf 39},~347~(1986)\relax
\relax
\bibitem{Jetsetb}
T.~Sj\"ostrand and M.~Bengtsson,
\newblock Comp. Phys. Comm.{} {\bf 43},~367~(1987)\relax
\relax
\bibitem{Jetsetc}
T.~Sj\"ostrand,
\newblock Comp. Phys. Comm.{} {\bf 82},~74~(1994)\relax
\relax
\bibitem{Briskin_vm}
G.~Briskin.
\newblock PhD Thesis , changing LUKFDI for vm production\relax
\relax
\bibitem{Jpsi_H1_dis}
\mbox{H1} Collaboration; S.~Aid {\em et al.},
\newblock Nucl. Phys. {\bf B}{} {\bf 468},~3~(1996)\relax
\relax
\bibitem{Jpsi_ZEUS}
\mbox{ZEUS} Collaboration; J. Breitweg~et al.,
\newblock Z. Phys. {\bf C}{} {\bf 75},~215~(1997).
\newblock Hep-ex/9704013\relax
\relax
\bibitem{Jung_Eilat}
H.~Jung,
\newblock {\em Modelling diffractive processes},
\newblock in {\em Proc. of the Workshop on Diffractive Interactions},
  ed.~A.~Levy.
\newblock Eilat, Israel, Feb. 18 - 23, 1995\relax
\relax
\bibitem{CDM}
L.~L\"onnblad,
\newblock Comp. Phys. Comm.{} {\bf 71},~15~(1992)\relax
\relax
\bibitem{LHA-generator}
E.~Boos {\em et al.}~(2001).
\newblock Hep-ph/0109068\relax
\relax
\bibitem{Ingelman_LEPTO65}
G.~Ingelman, A.~Edin, and J.~Rathsman,
\newblock Comp. Phys. Comm.{} {\bf 101},~108~(1997).
\newblock \href{http://www.arXiv.org/abs/hep-ph/9605286}{{\tt
  hep-ph/9605286}}\relax
\relax
\bibitem{PYTHIA_ppbar_color}
M.~H. Bengtsson,
\newblock Comp. Phys. Comm.{} {\bf 31},~323~(1984)\relax
\relax
\bibitem{DIFFVM}
B.~List,
\newblock {\em Diffraktive $J/\psi$ Produktion in Elektron - Proton St\"o\ss en
  am Speicherring HERA} (unpublished).
\newblock \mbox{Diploma thesis, Techn. Univ. Berlin, H1 note: H1-10/93-319},
  1993\relax
\relax
\bibitem{DIFFVM2}
B.~List and A.~Mastroberardino,
\newblock {\em Diffvm: A monte carlo generator for diffractive processes in e p
  scattering},
\newblock in {\em Proceedings of the Workshop on Monte Carlo generators for
  HERA physics}, eds.~A.~Doyle, G.~Grindhammer, G.~Ingelman, and H.~Jung, p.
  396.
\newblock \mbox{DESY}, Hamburg, 1999\relax
\relax
\bibitem{PYTHIAPSa}
T.~Sj\"ostrand,
\newblock Phys. Lett. {\bf B}{} {\bf 157},~321~(1985)\relax
\relax
\bibitem{PYTHIAPSb}
M.~Bengtsson, T.~Sj\"ostrand, and M.~van Zijl,
\newblock Z. Phys. {\bf C}{} {\bf 32},~67~(1986)\relax
\relax
\bibitem{H1_energyflow}
\mbox{H1} Collaboration; I. Abt~et al.,
\newblock Z. Phys. {\bf C}{} {\bf 63},~377~(1994).
\newblock \mbox{DESY 94-033}\relax
\relax
\bibitem{LEPTOPS}
M.~Bengtsson and T.~Sj\"ostrand,
\newblock Z. Phys. {\bf C}{} {\bf 37},~465~(1988)\relax
\relax
\bibitem{Bartels_qqg}
J.~Bartels,
\newblock {\em qqg jets in diffraction},
\newblock in {\em Proc. of the LISHEP workshop on diffractive physics},
  ed.~A.~Santoro.
\newblock Rio de Janeiro, Brazil, Feb 16 - 20, 1998\relax
\relax
\bibitem{Kohrsa}
H.~Kohrs,
\newblock {\em Direct and resolved pomeron},
\newblock in {\em Proc. of the Workshop on Deep Inelastic Scattering and QCD},
  eds.~J.~Laporte and Y.~Sirois.
\newblock 1995\relax
\relax
\bibitem{Kohrsb}
B.~Kniehl, H.~Kohrs, and G.~Kramer,
\newblock Z. Phys. {\bf C}{} {\bf 65},~657~(1995)\relax
\relax
\bibitem{Holtmann}
H.~Holtmann, G.~Levman, N.~Nikolaev, A.~Szczurek, and J.~Speth,
\newblock Phys. Lett. {\bf B}{} {\bf 338},~363~(1994)\relax
\relax
\bibitem{hztool2006}
B.~M. Waugh {\em et al.}~(2006).
\newblock \href{http://www.arXiv.org/abs/hep-ph/0605034}{{\tt
  hep-ph/0605034}}\relax
\relax
\end{mcbibliography}

\end{document}


